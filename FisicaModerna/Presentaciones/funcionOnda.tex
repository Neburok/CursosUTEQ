
	

	% --- Diapositiva 3a: Recordando las Ondas Clásicas (Parte 1) ---

	% --- Diapositiva 3b: Recordando las Ondas Clásicas (Parte 2) ---

	
	% --- Diapositiva 3c: Analogía Visual: De lo Clásico a lo Cuántico ---
	
	
	% --- Diapositiva 4: ¿Qué es la Función de Onda (Ψ)? ---

	% --- Diapositiva 5: Un Caso Especial: Estados Estacionarios ---

	% --- Diapositiva 6: La Gran Pregunta... ---

	
	% --- Diapositiva 7: La Interpretación de Born: Probabilidad ---

	
	% --- Diapositiva 8: Normalización ---
	\begin{frame}{Normalización}
		\begin{block}{La partícula tiene que estar en algún lugar}
			Si $|\Psi|^2$ representa una probabilidad, la probabilidad total de encontrar la partícula en \textit{todo el universo} debe ser del 100\% (o 1).
			\vspace{0.5em}
			Esto impone una condición matemática fundamental:
			\begin{equation*}
				\int_{\text{todo el espacio}} |\Psi|^2 dV = 1
			\end{equation*}
			Cualquier función de onda físicamente válida debe estar \textbf{normalizada}.
		\end{block}
	\end{frame}
	
	% --- Diapositiva 9: La Ley Fundamental del Mundo Cuántico ---
	\begin{frame}{La Ley Fundamental del Mundo Cuántico}
		\begin{columns}
			\column{0.5\textwidth}
			\begin{block}{Mecánica Clásica}
				Conociendo las fuerzas, \textbf{F = ma} nos dice cómo se moverá un objeto.
			\end{block}
			\column{0.5\textwidth}
			\begin{block}{Mecánica Cuántica}
				Conociendo el entorno (la energía potencial $U(x)$), necesitamos una ley que nos diga cómo será la función de onda $\psi(x)$.
			\end{block}
		\end{columns}
		\vfill
		\centering
		Esa ley es la \textbf{Ecuación de Schrödinger}.
		\begin{figure}
			\includegraphics[width=0.3\textwidth]{example-image} % Reemplaza con imagen de Schrödinger
			\caption{Erwin Schrödinger (p. 1364).}
		\end{figure}
	\end{frame}
	
	% --- Diapositiva 10: La Ecuación de Schrödinger ---
	\begin{frame}{La Ecuación de Schrödinger}
		\centering\textbf{(Independiente del tiempo, 1D)}\medskip
		\begin{beamercolorbox}[sep=0.3cm,center,wd=\textwidth]{block body}
			\Huge
			\begin{equation*}
				-\frac{\hbar^2}{2m} \frac{d^2\psi(x)}{dx^2} + U(x)\psi(x) = E\psi(x)
			\end{equation*}
		\end{beamercolorbox}
		\medskip
		\centering (Ecuación 39.18, p. 1364)
	\end{frame}
	
	% --- Diapositiva 11: Anatomía de la Ecuación ---
	\begin{frame}{Anatomía de la Ecuación}
		Desglosando sus partes:
		\begin{align*}
			\underbrace{-\frac{\hbar^2}{2m} \frac{d^2\psi}{dx^2}}_{\text{Término de Energía Cinética}} + \underbrace{U(x)\psi(x)}_{\text{Término de Energía Potencial}} &= \underbrace{E\psi(x)}_{\text{Energía Total del estado}}
		\end{align*}
		\vspace{1em}
		La ecuación dice: \textbf{(Energía Cinética + Energía Potencial)$\psi$ = (Energía Total)$\psi$}
	\end{frame}
	
	% --- Diapositiva 12: La Magia de la Ecuación de Schrödinger ---
	\begin{frame}{La Magia de la Ecuación de Schrödinger}
		\begin{block}{¿Por qué es tan importante?}
			Resolver la ecuación para un potencial $U(x)$ dado nos da dos cosas:
			\begin{enumerate}
				\item Las \textbf{funciones de onda permitidas, $\psi(x)$}, que describen el estado de la partícula.
				\item Los \textbf{niveles de energía permitidos, $E$}, para cada estado.
			\end{enumerate}
		\end{block}
		\pause
		\begin{alertblock}{}
			\centering
			\textbf{La cuantización de la energía no se postula, ¡sino que surge como una consecuencia natural de la naturaleza ondulatoria de la materia!}
		\end{alertblock}
	\end{frame}
	
	% --- Diapositiva 13: Resumen de la Clase ---
	\begin{frame}{Resumen de la Clase}
		\begin{enumerate}
			\item Las partículas se describen por una \textbf{función de onda ($\Psi$)}, que contiene toda su información.
			\item El significado físico es la \textbf{densidad de probabilidad, $|\Psi|^2$}, que nos dice dónde es más probable encontrar la partícula.
			\item La \textbf{Ecuación de Schrödinger} es la ley fundamental que nos permite encontrar la función de onda y la energía de un sistema.
			\item Resolverla demuestra que la \textbf{energía está cuantizada} de forma natural.
		\end{enumerate}
	\end{frame}
	
	% --- Diapositiva 14: Próximos Pasos ---
	\begin{frame}{Próximos Pasos}
		\begin{block}{¿Qué sigue?}
			En la próxima clase, aplicaremos la Ecuación de Schrödinger al sistema cuántico más simple:
			\vspace{1em}
			\centering\textbf{\Large La partícula en una caja.}
			\vspace{1em}
			Veremos cómo resolver la ecuación paso a paso y cómo aparecen los niveles de energía y las funciones de onda cuantizadas.
		\end{block}
	\end{frame}
	
\end{document}

 (\usetheme{Madrid}), los colores, y añadir más elementos visuales para hacerla aún más atractiva.