\documentclass[12pt,aspectratio=169]{beamer}

% Tema y configuración
\usetheme{Madrid}
\usecolortheme{seahorse}
\usefonttheme{professionalfonts}

% Paquetes necesarios
\usepackage[utf8]{inputenc}
\usepackage[spanish]{babel}
\usepackage{amsmath,amssymb,amsfonts}
\usepackage{graphicx}
\usepackage{tikz}
\usepackage{pgfplots}
\usepackage{xcolor}
\usepackage{multicol}
\usepackage{textcomp}
\usepackage{siunitx}

% Configuración de TikZ
\usetikzlibrary{shapes,arrows,positioning,decorations.pathmorphing}
\pgfplotsset{compat=1.17}

% Colores personalizados
\definecolor{azuloscuro}{RGB}{25,25,112}
\definecolor{rojoclaro}{RGB}{220,20,60}
\definecolor{verdeclaro}{RGB}{50,205,50}
\definecolor{naranjaclaro}{RGB}{255,140,0}

% Configuración de fuentes más grandes
\setbeamerfont{title}{size=\huge,series=\bfseries}
\setbeamerfont{subtitle}{size=\Large}
\setbeamerfont{frametitle}{size=\Large,series=\bfseries}
\setbeamerfont{framesubtitle}{size=\large}
\setbeamerfont{normal text}{size=\large}
\setbeamerfont{caption}{size=\large}

% Configuración de ecuaciones grandes
\everymath{\displaystyle}

% Información del documento
\title[Espectros Atómicos]{Los Espectros Atómicos y la Inestabilidad de los Átomos Clásicos}
\subtitle{Física Moderna - Capítulo 3}
\author{Profesor}
\institute{Universidad}
\date{\today}

\begin{document}

% ==================== SLIDE 1: PORTADA ====================
\begin{frame}
    \titlepage
    \vspace{1cm}
    \begin{center}
        \Large
        \textbf{Duración:} 100 minutos \\
        \textbf{Nivel:} Ingeniería/Licenciatura
    \end{center}
\end{frame}

% ==================== SLIDE 2: OBJETIVOS ====================
\begin{frame}{Objetivos de Aprendizaje}
    \vspace{0.5cm}
    \begin{center}
        \huge \textbf{¿Qué aprenderemos hoy?}
    \end{center}
    
    \vspace{1cm}
    \begin{itemize}
        \item[\textbullet] \Large \textbf{Comprender} las limitaciones del modelo clásico
        \vspace{0.5cm}
        \item[\textbullet] \Large \textbf{Analizar} la ecuación empírica de Balmer
        \vspace{0.5cm}
        \item[\textbullet] \Large \textbf{Aplicar} los postulados de Bohr
        \vspace{0.5cm}
        \item[\textbullet] \Large \textbf{Evaluar} la revolución cuántica
    \end{itemize}
\end{frame}

% ==================== SLIDE 3: CRONOLOGÍA ====================
\begin{frame}{Contexto Histórico}
    \vspace{0.5cm}
    \begin{center}
        \huge \textbf{Los Descubrimientos Clave}
    \end{center}
    
    \vspace{1cm}
    \begin{center}
        \begin{tikzpicture}[scale=1.2]
            % Línea de tiempo
            \draw[thick,->] (0,0) -- (12,0);
            
            % Marcadores de años
            \foreach \x/\year in {2/1885, 5/1900, 8/1911, 10/1913, 12/1920s} {
                \draw (\x,0) -- (\x,0.3);
                \node[below] at (\x,-0.3) {\Large \textbf{\year}};
            }
            
            % Eventos
            \node[above,text width=2.5cm,align=center] at (2,0.5) {\large \textbf{Balmer} \\ Fórmula empírica};
            \node[above,text width=2.5cm,align=center] at (8,0.5) {\large \textbf{Rutherford} \\ Modelo planetario};
            \node[above,text width=2.5cm,align=center] at (10,0.5) {\large \textbf{Bohr} \\ Postulados cuánticos};
            \node[above,text width=2.5cm,align=center] at (12,0.5) {\large \textbf{Mecánica} \\ \textbf{Cuántica}};
        \end{tikzpicture}
    \end{center}
\end{frame}

% ==================== SLIDE 4: ESPECTROS CONTINUOS VS DISCRETOS ====================
\begin{frame}{El Misterio de las Líneas Espectrales}
    \begin{center}
        \huge \textbf{¿Por qué líneas discretas?}
    \end{center}
    
    \vspace{1cm}
    \begin{center}
        \begin{tikzpicture}[scale=1.5]
            % Espectro continuo
            \fill[left color=red, right color=violet] (0,2) rectangle (8,2.8);
            \node[left] at (0,2.4) {\Large \textbf{Continuo}};
            
            % Espectro de líneas
            \fill[black] (0,0) rectangle (8,0.8);
            \foreach \x/\color in {1.5/red, 3/cyan, 4.5/blue, 6/violet} {
                \fill[\color] (\x,0) rectangle (\x+0.2,0.8);
            }
            \node[left] at (0,0.4) {\Large \textbf{Hidrógeno}};
            
            % Etiquetas
            \node[below] at (1.6,-0.3) {\large H$_\alpha$};
            \node[below] at (3.1,-0.3) {\large H$_\beta$};
            \node[below] at (4.6,-0.3) {\large H$_\gamma$};
            \node[below] at (6.1,-0.3) {\large H$_\delta$};
        \end{tikzpicture}
    \end{center}
    
    \vspace{1cm}
    \begin{center}
        \LARGE \textcolor{rojoclaro}{\textbf{¿Por qué solo frecuencias específicas?}}
    \end{center}
\end{frame}

% ==================== SLIDE 5: ECUACIÓN DE BALMER ====================
\begin{frame}{La Ecuación Empírica de Balmer}
    \begin{center}
        \huge \textbf{El Patrón Matemático (1885)}
    \end{center}
    
    \vspace{1cm}
    \begin{center}
        \colorbox{azuloscuro!10}{\parbox{0.8\textwidth}{
            \begin{center}
                \Huge \textcolor{azuloscuro}{
                    $$\frac{1}{\lambda} = R_H\left(\frac{1}{n_1^2} - \frac{1}{n_2^2}\right)$$
                }
            \end{center}
        }}
    \end{center}
    
    \vspace{1cm}
    \begin{center}
        \begin{tabular}{cl}
            \Large $\lambda$ & \Large Longitud de onda \\[0.3cm]
            \Large $R_H$ & \Large Constante de Rydberg = $1.097 \times 10^7$ m$^{-1}$ \\[0.3cm]
            \Large $n_1, n_2$ & \Large Números enteros ($n_2 > n_1$)
        \end{tabular}
    \end{center}
\end{frame}

% ==================== SLIDE 6: SERIES ESPECTRALES ====================
\begin{frame}{Series Espectrales del Hidrógeno}
    \vspace{0.5cm}
    \begin{center}
        \huge \textbf{Clasificación por $n_1$}
    \end{center}
    
    \vspace{1cm}
    \begin{center}
        \begin{tikzpicture}[scale=1.3]
            % Diagrama de niveles
            \foreach \n/\y/\energy in {1/1/-13.6, 2/2.5/-3.4, 3/3.5/-1.51, 4/4.2/-0.85, 5/4.7/0} {
                \draw[thick] (0,\y) -- (8,\y);
                \node[left] at (0,\y) {\Large $n=\n$};
                \node[right] at (8,\y) {\Large \energy eV};
            }
            
            % Transiciones de Lyman (n1=1)
            \draw[thick,red,->] (1.5,2.5) -- (1.5,1.2);
            \draw[thick,red,->] (2,3.5) -- (2,1.2);
            \node[below] at (1.75,0.5) {\Large \textcolor{red}{\textbf{Lyman (UV)}}};
            
            % Transiciones de Balmer (n1=2)
            \draw[thick,blue,->] (4.5,3.5) -- (4.5,2.7);
            \draw[thick,blue,->] (5,4.2) -- (5,2.7);
            \node[below] at (4.75,0.5) {\Large \textcolor{blue}{\textbf{Balmer (Vis)}}};
            
            % Transiciones de Paschen (n1=3)
            \draw[thick,green,->] (6.5,4.2) -- (6.5,3.7);
            \draw[thick,green,->] (7,4.7) -- (7,3.7);
            \node[below] at (6.75,0.5) {\Large \textcolor{green}{\textbf{Paschen (IR)}}};
        \end{tikzpicture}
    \end{center}
\end{frame}

% ==================== SLIDE 7: CÁLCULO EJEMPLO H-ALPHA ====================
\begin{frame}{Ejemplo: Cálculo de H$_\alpha$}
    \begin{center}
        \huge \textbf{Línea Roja del Hidrógeno}
    \end{center}
    
    \vspace{0.5cm}
    \begin{center}
        \Large \textbf{Transición:} $n = 3 \rightarrow n = 2$
    \end{center}
    
    \vspace{1cm}
    \begin{align}
        \frac{1}{\lambda} &= R_H\left(\frac{1}{2^2} - \frac{1}{3^2}\right) \\[0.5cm]
        \frac{1}{\lambda} &= 1.097 \times 10^7 \left(\frac{1}{4} - \frac{1}{9}\right) \\[0.5cm]
        \frac{1}{\lambda} &= 1.097 \times 10^7 \times \frac{5}{36} \\[0.5cm]
        \lambda &= \boxed{656.3 \text{ nm}}
    \end{align}
    
    \vspace{0.5cm}
    \begin{center}
        \LARGE \textcolor{red}{\textbf{¡Luz roja visible!}}
    \end{center}
\end{frame}

% ==================== SLIDE 8: PROBLEMA CLÁSICO ====================
\begin{frame}{Crisis del Modelo Clásico}
    \begin{center}
        \huge \textbf{¿Por qué no colapsan los átomos?}
    \end{center}
    
    \vspace{1cm}
    \begin{center}
        \begin{tikzpicture}[scale=1.5]
            % Núcleo
            \fill[red] (0,0) circle (0.3);
            \node at (0,0) {\textbf{+}};
            
            % Órbitas con electrón colapsando
            \foreach \r/\opacity in {2/1, 1.5/0.7, 1/0.4, 0.5/0.2} {
                \draw[dashed,opacity=\opacity] (0,0) circle (\r);
            }
            
            % Electrón
            \fill[blue] (2,0) circle (0.15);
            \node at (2,0) {\textbf{-}};
            
            % Flecha de colapso
            \draw[thick,red,->] (1.5,0) arc (0:180:0.75);
            \node[above] at (0.75,0.5) {\Large \textcolor{red}{\textbf{Colapso}}};
        \end{tikzpicture}
    \end{center}
    
    \vspace{1cm}
    \begin{center}
        \LARGE \textcolor{rojoclaro}{\textbf{Predicción clásica: Colapso en $\sim 10^{-11}$ s}} \\[0.5cm]
        \LARGE \textcolor{verdeclaro}{\textbf{Realidad: Los átomos son estables}}
    \end{center}
\end{frame}

% ==================== SLIDE 9: ECUACIONES CLÁSICAS ====================
\begin{frame}{Las Ecuaciones del Problema Clásico}
    \vspace{0.5cm}
    \begin{center}
        \huge \textbf{Electromagnetismo Clásico}
    \end{center}
    
    \vspace{1cm}
    \begin{center}
        \colorbox{rojoclaro!10}{\parbox{0.9\textwidth}{
            \begin{center}
                \Large \textbf{Aceleración centrípeta:}
                \Huge $$a_c = \frac{v^2}{r}$$
            \end{center}
        }}
    \end{center}
    
    \vspace{1cm}
    \begin{center}
        \colorbox{rojoclaro!10}{\parbox{0.9\textwidth}{
            \begin{center}
                \Large \textbf{Potencia radiada (Larmor):}
                \Huge $$P = \frac{2e^2a^2}{3c^3}$$
            \end{center}
        }}
    \end{center}
    
    \vspace{1cm}
    \begin{center}
        \LARGE \textcolor{rojoclaro}{\textbf{Resultado: $t_{\text{colapso}} \approx 1.6 \times 10^{-11}$ s}}
    \end{center}
\end{frame}

% ==================== SLIDE 10: POSTULADOS DE BOHR ====================
\begin{frame}{Los Postulados de Bohr (1913)}
    \begin{center}
        \huge \textbf{La Revolución Cuántica}
    \end{center}
    
    \vspace{1cm}
    \begin{center}
        \colorbox{azuloscuro!10}{\parbox{0.95\textwidth}{
            \Large \textbf{Postulado 1:} Los electrones orbitan en \textbf{órbitas estacionarias} sin radiar
        }}
    \end{center}
    
    \vspace{0.8cm}
    \begin{center}
        \colorbox{verdeclaro!10}{\parbox{0.95\textwidth}{
            \Large \textbf{Postulado 2:} Solo están permitidas órbitas con
            \Huge $$L = mvr = n\hbar$$
        }}
    \end{center}
    
    \vspace{0.8cm}
    \begin{center}
        \colorbox{naranjaclaro!10}{\parbox{0.95\textwidth}{
            \Large \textbf{Postulado 3:} La radiación ocurre en transiciones:
            \Huge $$h\nu = E_i - E_f$$
        }}
    \end{center}
\end{frame}

% ==================== SLIDE 11: DESARROLLO MATEMÁTICO PASO 1 ====================
\begin{frame}{Desarrollo Matemático - Paso 1}
    \begin{center}
        \huge \textbf{Equilibrio de Fuerzas}
    \end{center}
    
    \vspace{1cm}
    \begin{center}
        \colorbox{azuloscuro!10}{\parbox{0.9\textwidth}{
            \begin{center}
                \Large \textbf{Fuerza eléctrica = Fuerza centrípeta}
                \vspace{0.5cm}
                \Huge $$\frac{ke^2}{r^2} = \frac{mv^2}{r}$$
            \end{center}
        }}
    \end{center}
    
    \vspace{1cm}
    \begin{center}
        \colorbox{verdeclaro!10}{\parbox{0.9\textwidth}{
            \begin{center}
                \Large \textbf{Condición de cuantización:}
                \vspace{0.5cm}
                \Huge $$mvr = n\hbar \quad \Rightarrow \quad v = \frac{n\hbar}{mr}$$
            \end{center}
        }}
    \end{center}
\end{frame}

% ==================== SLIDE 12: DESARROLLO MATEMÁTICO PASO 2 ====================
\begin{frame}{Desarrollo Matemático - Paso 2}
    \begin{center}
        \huge \textbf{Radios Permitidos}
    \end{center}
    
    \vspace{1cm}
    \begin{center}
        \Large \textbf{Sustituyendo $v = \frac{n\hbar}{mr}$ en la ecuación de equilibrio:}
    \end{center}
    
    \vspace{1cm}
    \begin{align}
        \frac{ke^2}{r^2} &= \frac{m}{r}\left(\frac{n\hbar}{mr}\right)^2 \\[0.8cm]
        \frac{ke^2}{r^2} &= \frac{n^2\hbar^2}{mr^3} \\[0.8cm]
        r_n &= \frac{n^2\hbar^2}{mke^2} = n^2 a_0
    \end{align}
    
    \vspace{1cm}
    \begin{center}
        \colorbox{verdeclaro!20}{\parbox{0.8\textwidth}{
            \begin{center}
                \LARGE \textbf{Radio de Bohr:} $a_0 = 0.529 \times 10^{-10}$ m
            \end{center}
        }}
    \end{center}
\end{frame}

% ==================== SLIDE 13: ENERGÍAS CUANTIZADAS ====================
\begin{frame}{Energías Cuantizadas}
    \begin{center}
        \huge \textbf{Los "Escalones" de Energía}
    \end{center}
    
    \vspace{1cm}
    \begin{center}
        \colorbox{azuloscuro!10}{\parbox{0.9\textwidth}{
            \begin{center}
                \Large \textbf{Energía total del electrón:}
                \vspace{0.5cm}
                \Huge $$E_n = -\frac{mke^4}{2\hbar^2n^2} = -\frac{13.6 \text{ eV}}{n^2}$$
            \end{center}
        }}
    \end{center}
    
    \vspace{1cm}
    \begin{center}
        \begin{tabular}{cc}
            \Large $n = 1$ & \Large $E_1 = -13.6$ eV \\[0.3cm]
            \Large $n = 2$ & \Large $E_2 = -3.4$ eV \\[0.3cm]
            \Large $n = 3$ & \Large $E_3 = -1.51$ eV \\[0.3cm]
            \Large $n = \infty$ & \Large $E_\infty = 0$ eV
        \end{tabular}
    \end{center}
    
    \vspace{0.5cm}
    \begin{center}
        \LARGE \textcolor{rojoclaro}{\textbf{Energía de ionización = 13.6 eV}}
    \end{center}
\end{frame}

% ==================== SLIDE 14: CONSTANTE DE RYDBERG DERIVADA ====================
\begin{frame}{Derivación de la Constante de Rydberg}
    \begin{center}
        \huge \textbf{Teoría $\rightarrow$ Experimento}
    \end{center}
    
    \vspace{1cm}
    \begin{center}
        \Large \textbf{Frecuencia de transición:}
    \end{center}
    
    \vspace{0.5cm}
    \begin{align}
        \nu &= \frac{E_i - E_f}{h} \\[0.5cm]
        \nu &= \frac{mke^4}{4\pi\hbar^3} \left(\frac{1}{n_f^2} - \frac{1}{n_i^2}\right) \\[0.5cm]
        \frac{1}{\lambda} &= \frac{\nu}{c} = R_H \left(\frac{1}{n_f^2} - \frac{1}{n_i^2}\right)
    \end{align}
    
    \vspace{1cm}
    \begin{center}
        \colorbox{verdeclaro!20}{\parbox{0.9\textwidth}{
            \begin{center}
                \LARGE $$R_H = \frac{mke^4}{4\pi\hbar^3c} = 1.097 \times 10^7 \text{ m}^{-1}$$
            \end{center}
        }}
    \end{center}
    
    \vspace{0.5cm}
    \begin{center}
        \LARGE \textcolor{azuloscuro}{\textbf{¡Coincide exactamente con el experimento!}}
    \end{center}
\end{frame}

% ==================== SLIDE 15: DIAGRAMA DE NIVELES COMPLETO ====================
\begin{frame}{Diagrama de Niveles de Energía}
    \vspace{0.5cm}
    \begin{center}
        \huge \textbf{Transiciones y Series}
    \end{center}
    
    \vspace{0.5cm}
    \begin{center}
        \begin{tikzpicture}[scale=1.4]
            % Niveles de energía
            \foreach \n/\y/\energy in {1/1/-13.6, 2/2.8/-3.4, 3/4.0/-1.51, 4/4.8/-0.85, 5/5.4/0} {
                \draw[thick] (0,\y) -- (9,\y);
                \node[left] at (-0.5,\y) {\Large $n=\n$};
                \node[right] at (9.5,\y) {\Large \energy eV};
            }
            
            % Serie de Lyman (UV)
            \draw[thick,violet,->] (1.5,2.8) -- (1.5,1.2);
            \draw[thick,violet,->] (1.8,4.0) -- (1.8,1.2);
            \draw[thick,violet,->] (2.1,4.8) -- (2.1,1.2);
            \node[below] at (1.8,0.3) {\Large \textcolor{violet}{\textbf{Lyman}}};
            \node[below] at (1.8,0) {\Large \textcolor{violet}{\textbf{(UV)}}};
            
            % Serie de Balmer (Visible)
            \draw[thick,red,->] (4.5,4.0) -- (4.5,3.0);
            \draw[thick,cyan,->] (4.8,4.8) -- (4.8,3.0);
            \draw[thick,blue,->] (5.1,5.4) -- (5.1,3.0);
            \node[below] at (4.8,0.3) {\Large \textcolor{red}{\textbf{Balmer}}};
            \node[below] at (4.8,0) {\Large \textcolor{red}{\textbf{(Visible)}}};
            
            % Serie de Paschen (IR)
            \draw[thick,orange,->] (7.2,4.8) -- (7.2,4.2);
            \draw[thick,orange,->] (7.5,5.4) -- (7.5,4.2);
            \node[below] at (7.35,0.3) {\Large \textcolor{orange}{\textbf{Paschen}}};
            \node[below] at (7.35,0) {\Large \textcolor{orange}{\textbf{(IR)}}};
            
            % Etiquetas de colores específicos
            \node[right] at (5.2,3.5) {\large H$_\alpha$ (656 nm)};
            \node[right] at (5.2,3.9) {\large H$_\beta$ (486 nm)};
        \end{tikzpicture}    \end{center}
\end{frame}

% ==================== SLIDE 16: EJEMPLO PRÁCTICO H-BETA ====================
\begin{frame}{Ejemplo Práctico: H$_\beta$}
        \huge \textbf{Línea Azul-Verde}
    \end{center}
    
    \vspace{0.5cm}
    \begin{center}
        \Large \textbf{Transición:} $n = 4 \rightarrow n = 2$    \end{center}
    
    \vspace{1cm}
    \begin{align}
        \frac{1}{\lambda} &= R_H\left(\frac{1}{2^2} - \frac{1}{4^2}\right) \\[0.5cm]
        \frac{1}{\lambda} &= 1.097 \times 10^7 \left(\frac{1}{4} - \frac{1}{16}\right) \\[0.5cm]
        \frac{1}{\lambda} &= 1.097 \times 10^7 \times \frac{3}{16} \\[0.5cm]
        \lambda &= \boxed{486.1 \text{ nm}}
    \end{align}
    
    \vspace{1cm}
    \begin{center}
        \begin{tikzpicture}
            \fill[cyan] (0,0) rectangle (6,1);
            \node at (3,0.5) {\Large \textbf{486.1 nm (H$_\beta$)}};
        \end{tikzpicture}
    \end{center}
\end{frame}

% ==================== SLIDE 17: APLICACIONES MODERNAS ====================
\begin{frame}{Aplicaciones Modernas}    \begin{center}
        \huge \textbf{De la Teoría a la Tecnología}
    \end{center}
    
    \vspace{1cm}
    \begin{center}
        \begin{tikzpicture}[scale=1.2]
            % Espectroscopía
            \node[draw,rectangle,minimum width=3cm,minimum height=1.5cm,fill=azuloscuro!20] at (0,3) {};
            \node at (0,3) {\Large \textbf{Espectroscopía}};
            \node at (0,2.2) {\large Análisis químico};
            
            % Astrofísica
            \node[draw,rectangle,minimum width=3cm,minimum height=1.5cm,fill=verdeclaro!20] at (5,3) {};
            \node at (5,3) {\Large \textbf{Astrofísica}};
            \node at (5,2.2) {\large Composición estelar};
            
            % Láseres
            \node[draw,rectangle,minimum width=3cm,minimum height=1.5cm,fill=rojoclaro!20] at (0,0) {};
            \node at (0,0) {\Large \textbf{Láseres}};
            \node at (0,-0.8) {\large Emisión estimulada};
            
            % Medicina
            \node[draw,rectangle,minimum width=3cm,minimum height=1.5cm,fill=naranjaclaro!20] at (5,0) {};
            \node at (5,0) {\Large \textbf{Medicina}};
            \node at (5,-0.8) {\large Resonancia magnética};
        \end{tikzpicture}
    \end{center}
    
    \vspace{1cm}    \begin{center}
        \LARGE \textbf{Los espectros atómicos son la base de tecnologías modernas}
    \end{center}
\end{frame}

% ==================== SLIDE 18: LIMITACIONES ====================
\begin{frame}{Limitaciones del Modelo de Bohr}
    \begin{center}
        \huge \textbf{¿Por qué necesitamos mecánica cuántica?}
    \end{center}
    
    \vspace{1cm}
    \begin{center}
        \begin{tikzpicture}[scale=1.3]
            % Éxitos
            \node[draw,rectangle,minimum width=4cm,minimum height=3cm,fill=verdeclaro!20] at (-3,0) {};
            \node[above] at (-3,1.2) {\Large \textbf{✓ ÉXITOS}};
            \node[text width=3.5cm,align=center] at (-3,-0.2) {
                \large 
                • Espectros de H \\
                • Constante de Rydberg \\
                • Estabilidad atómica \\
                • Cuantización
            };
            
            % Limitaciones
            \node[draw,rectangle,minimum width=4cm,minimum height=3cm,fill=rojoclaro!20] at (3,0) {};
            \node[above] at (3,1.2) {\Large \textbf{✗ LIMITACIONES}};
            \node[text width=3.5cm,align=center] at (3,-0.2) {
                \large 
                • Átomos multielectrónicos \\
                • Intensidades de líneas \\
                • Estructura fina \\
                • Base teórica ad hoc
            };
        \end{tikzpicture}
    \end{center>
    
    \vspace{1cm}    \begin{center}
        \LARGE \textcolor{azuloscuro}{\textbf{Solución: Mecánica Cuántica Completa (1920s)}}
    \end{center}
\end{frame}

% ==================== SLIDE 19: EJERCICIO PRÁCTICO ====================
\begin{frame}{Ejercicio en Clase}    \begin{center}
        \huge \textbf{¡Vamos a calcular!}
    \end{center}
    
    \vspace{1cm}
    \begin{center}
        \colorbox{naranjaclaro!20}{\parbox{0.9\textwidth}{
            \begin{center}
                \Large \textbf{Problema:} Una línea espectral del hidrógeno tiene $\lambda = 434.0$ nm.
                
                \vspace{0.5cm}
                \Large \textbf{Encuentra:}
                \begin{itemize}
                    \item[\textbullet] \Large ¿A qué serie pertenece?
                    \item[\textbullet] \Large ¿Cuál es la transición ($n_i \rightarrow n_f$)?
                    \item[\textbullet] \Large ¿Qué color observamos?
                \end{itemize}
            \end{center}
        }}
    \end{center>
    
    \vspace{1cm}
    \begin{center}
        \Large \textbf{Tiempo: 5 minutos}
        
        \vspace{0.5cm}
        \Large \textbf{Pista:} Usa $\frac{1}{\lambda} = R_H\left(\frac{1}{n_1^2} - \frac{1}{n_2^2}\right)$
    \end{center>
\end{frame>

% ==================== SLIDE 20: SOLUCIÓN DEL EJERCICIO ====================
\begin{frame}{Solución del Ejercicio}
    \begin{center}
        \huge \textbf{Paso a paso}
    \end{center>
    
    \vspace{0.5cm}
    \begin{align}
        \frac{1}{\lambda} &= \frac{1}{434.0 \times 10^{-9}} = 2.304 \times 10^6 \text{ m}^{-1} \\[0.5cm]
        \frac{2.304 \times 10^6}{1.097 \times 10^7} &= \frac{1}{n_1^2} - \frac{1}{n_2^2} = 0.210 \\[0.5cm]
        \text{Probando } n_1 = 2: \quad \frac{1}{4} - \frac{1}{n_2^2} &= 0.210 \\[0.5cm]
        \frac{1}{n_2^2} &= 0.25 - 0.210 = 0.040 \\[0.5cm]
        n_2^2 &= 25 \quad \Rightarrow \quad n_2 = 5
    \end{align}
    
    \vspace{0.5cm}
    \begin{center}
        \colorbox{verdeclaro!20}{\parbox{0.8\textwidth}{
            \begin{center}
                \LARGE \textbf{Respuesta:} \\
                \Large Serie de Balmer ($n_1 = 2$) \\
                \Large Transición: $n = 5 \rightarrow n = 2$ \\                \Large Color: Violeta (H$_\gamma$)
            \end{center}
        }}
    \end{center}
\end{frame}

% ==================== SLIDE 21: PROBLEMA INTEGRADOR ====================
\begin{frame}{Problema Integrador}
    \begin{center}
        \huge \textbf{Análisis Espectral Completo}
    \end{center>
    
    \vspace{0.5cm}
    \begin{center}
        \colorbox{azuloscuro!10}{\parbox{0.95\textwidth}{
            \begin{center}
                \Large \textbf{Un espectrógrafo registra estas líneas:}
                
                \vspace{0.5cm}
                \begin{tabular}{ll}
                    \Large Línea A: & \Large 656.3 nm (roja) \\
                    \Large Línea B: & \Large 486.1 nm (azul-verde) \\
                    \Large Línea C: & \Large 434.0 nm (violeta) \\
                    \Large Línea D: & \Large 410.2 nm (violeta)
                \end{tabular}
            \end{center>
        }}
    \end{center>
    
    \vspace{1cm}
    \begin{center}
        \Large \textbf{Tareas:}
        \begin{itemize}
            \item[\textbullet] \Large Identifica el elemento
            \item[\textbullet] \Large Determina las transiciones
            \item[\textbullet] \Large Predice la siguiente línea de la serie
        \end{itemize}
    \end{center>
\end{frame>

% ==================== SLIDE 22: CONEXIONES CONCEPTUALES ====================
\begin{frame}{Conexiones con Otros Temas}
    \begin{center}
        \huge \textbf{Integrando el Conocimiento}
    \end{center>
    
    \vspace{1cm}
    \begin{center}
        \begin{tikzpicture}[scale=1.4]
            % Tema central
            \node[draw,ellipse,minimum width=4cm,minimum height=2cm,fill=azuloscuro!20] at (0,0) {};
            \node at (0,0) {\Large \textbf{Espectros} \\ \Large \textbf{Atómicos}};
            
            % Temas previos
            \node[draw,rectangle,fill=verdeclaro!20] at (-4,2) {\large Cuerpo Negro};
            \node[draw,rectangle,fill=verdeclaro!20] at (-4,0) {\large Efecto Fotoeléctrico};
            \node[draw,rectangle,fill=verdeclaro!20] at (-4,-2) {\large Modelo Rutherford};
            
            % Temas posteriores
            \node[draw,rectangle,fill=naranjaclaro!20] at (4,2) {\large Dualidad Onda-Partícula};
            \node[draw,rectangle,fill=naranjaclaro!20] at (4,0) {\large Principio Incertidumbre};
            \node[draw,rectangle,fill=naranjaclaro!20] at (4,-2) {\large Ecuación Schrödinger};
            
            % Flechas
            \draw[thick,->] (-2.5,2) -- (-1.5,0.5);
            \draw[thick,->] (-2.5,0) -- (-1.5,0);
            \draw[thick,->] (-2.5,-2) -- (-1.5,-0.5);
            
            \draw[thick,->] (1.5,0.5) -- (2.5,2);
            \draw[thick,->] (1.5,0) -- (2.5,0);
            \draw[thick,->] (1.5,-0.5) -- (2.5,-2);
        \end{tikzpicture}
    \end{center}
\end{frame>

% ==================== SLIDE 23: PREGUNTAS DE REFLEXIÓN ====================
\begin{frame}{Preguntas de Reflexión}
    \begin{center}
        \huge \textbf{Pensamiento Crítico}
    \end{center>
    
    \vspace{1cm}
    \begin{center}
        \Large \textbf{Para discutir:}
    \end{center>
    
    \vspace{0.5cm}
    \begin{itemize}
        \item[\textbullet] \Large ¿Por qué los postulados de Bohr parecían tan revolucionarios en 1913?
        
        \vspace{0.5cm}
        \item[\textbullet] \Large Si el modelo funciona para hidrógeno, ¿por qué no es la teoría definitiva?
        
        \vspace{0.5cm}
        \item[\textbullet] \Large ¿Cómo usan los astrónomos los espectros para estudiar estrellas?
        
        \vspace{0.5cm}
        \item[\textbullet] \Large ¿Qué significa realmente que algo esté "cuantizado"?
    \end{itemize}
    
    \vspace{1cm}
    \begin{center}
        \LARGE \textcolor{azuloscuro}{\textbf{¡Discutamos en clase!}}
    \end{center}
\end{frame>

% ==================== SLIDE 24: PARA RECORDAR ====================
\begin{frame}{Conceptos Clave para Recordar}
    \begin{center}
        \huge \textbf{Los Puntos Esenciales}
    \end{center>
    
    \vspace{1cm}
    \begin{center}
        \colorbox{azuloscuro!10}{\parbox{0.9\textwidth}{
            \Large \textbf{1. CUANTIZACIÓN:} Solo ciertos valores discretos están permitidos
        }}
    \end{center>
    
    \vspace{0.5cm}
    \begin{center}
        \colorbox{verdeclaro!10}{\parbox{0.9\textwidth}{
            \Large \textbf{2. ESTABILIDAD:} Los postulados cuánticos resuelven problemas clásicos
        }}
    \end{center>
    
    \vspace{0.5cm}
    \begin{center}
        \colorbox{rojoclaro!10}{\parbox{0.9\textwidth}{
            \Large \textbf{3. ESPECTROS:} Evidencia directa de niveles de energía cuantizados
        }}
    \end{center>
    
    \vspace{0.5cm}
    \begin{center}
        \colorbox{naranjaclaro!10}{\parbox{0.9\textwidth}{
            \Large \textbf{4. TRANSICIÓN:} El modelo de Bohr como puente histórico
        }}
    \end{center>
    
    \vspace{1cm}
    \begin{center}
        \LARGE \textbf{Ecuación fundamental:} $\boxed{E_n = -\frac{13.6 \text{ eV}}{n^2}}$
    \end{center>
\end{frame>

% ==================== SLIDE 25: PRÓXIMA CLASE ====================
\begin{frame}{Próxima Clase}
    \begin{center}
        \huge \textbf{¿Qué sigue?}
    \end{center>
    
    \vspace{1cm}
    \begin{center}
        \begin{tikzpicture}[scale=1.3]
            % Bohr (actual)
            \node[draw,circle,minimum width=2.5cm,fill=azuloscuro!20] at (-3,0) {};
            \node at (-3,0) {\Large \textbf{Modelo} \\ \Large \textbf{de Bohr}};
            
            % Flecha
            \draw[thick,->,line width=3pt] (-1.5,0) -- (1.5,0);
            \node[above] at (0,0.5) {\Large \textbf{Evolución}};
            
            % Mecánica Cuántica (siguiente)
            \node[draw,circle,minimum width=2.5cm,fill=verdeclaro!20] at (3,0) {};
            \node at (3,0) {\Large \textbf{Mecánica} \\ \Large \textbf{Cuántica}};
        \end{tikzpicture>
    \end{center>
    
    \vspace{1cm}
    \begin{center}
        \Large \textbf{Temas por venir:}
        \begin{itemize}
            \item[\textbullet] \Large Dualidad onda-partícula de De Broglie
            \item[\textbullet] \Large Principio de incertidumbre de Heisenberg
            \item[\textbullet] \Large Ecuación de Schrödinger
            \item[\textbullet] \Large Orbitales atómicos modernos
        \end{itemize}
    \end{center}
\end{frame}

% ==================== SLIDE 26: RECURSOS ADICIONALES ====================
\begin{frame}{Recursos para Estudiar}
    \begin{center}
        \huge \textbf{Para Profundizar}
    \end{center}
    
    \vspace{1cm}
    \begin{center}
        \begin{tikzpicture}[scale=1.2]
            % Libros
            \node[draw,rectangle,minimum width=3cm,minimum height=2cm,fill=azuloscuro!20] at (-3,1) {};
            \node at (-3,1) {\Large \textbf{📚 Libros}};
            \node[text width=3cm,align=center] at (-3,0) {\large Eisberg \& Resnick \\ Cap. 4};
            
            % Videos
            \node[draw,rectangle,minimum width=3cm,minimum height=2cm,fill=verdeclaro!20] at (0,1) {};
            \node at (0,1) {\Large \textbf{🎥 Videos}};
            \node[text width=3cm,align=center] at (0,0) {\large MIT OCW \\ Khan Academy};
            
            % Simulaciones
            \node[draw,rectangle,minimum width=3cm,minimum height=2cm,fill=rojoclaro!20] at (3,1) {};
            \node at (3,1) {\Large \textbf{💻 Sims}};
            \node[text width=3cm,align=center] at (3,0) {\large PhET Colorado \\ NIST Database};
        \end{tikzpicture}
    \end{center}
    
    \vspace{1cm}
    \begin{center}
        \LARGE \textbf{¡Experimenten con las simulaciones!}
    \end{center>
\end{frame>

% ==================== SLIDE 27: EVALUACIÓN ====================
\begin{frame}{Evaluación}
    \begin{center}
        \huge \textbf{¿Cómo serán evaluados?}
    \end{center}
    
    \vspace{1cm}
    \begin{center}
        \colorbox{azuloscuro!10}{\parbox{0.9\textwidth}{
            \begin{center}
                \Large \textbf{Examen Parcial - Próxima Semana}
                
                \vspace{0.5cm}
                \large Incluirá:
                \begin{itemize}
                    \item[\textbullet] \large Cálculos con la ecuación de Balmer
                    \item[\textbullet] \large Identificación de series espectrales
                    \item[\textbullet] \large Postulados de Bohr y su aplicación
                    \item[\textbullet] \large Energías de transición
                    \item[\textbullet] \large Problema integrador de espectroscopía
                \end{itemize}
            \end{center}
        }}
    \end{center>
    
    \vspace{1cm}
    \begin{center}
        \LARGE \textcolor{rojoclaro}{\textbf{¡Estudien las ecuaciones y sus significados!}}
    \end{center>
\end{frame>

% ==================== SLIDE 28: RESUMEN FINAL ====================
\begin{frame}{Resumen de la Clase}
    \begin{center}
        \huge \textbf{Un Viaje Extraordinario}
    \end{center>
    
    \vspace{0.5cm}
    \begin{center}
        \Large \textbf{Hoy hemos visto cómo:}
    \end{center>
    
    \vspace{0.5cm}
    \begin{itemize}
        \item[\textbullet] \Large La física clásica \textcolor{rojoclaro}{\textbf{falló}} al explicar los átomos
        
        \vspace{0.3cm}
        \item[\textbullet] \Large Balmer encontró un \textcolor{azuloscuro}{\textbf{patrón matemático}} sin teoría
        
        \vspace{0.3cm}
        \item[\textbullet] \Large Bohr introdujo la \textcolor{verdeclaro}{\textbf{cuantización}} como solución
        
        \vspace{0.3cm}
        \item[\textbullet] \Large Los postulados explicaron \textcolor{naranjaclaro}{\textbf{perfectamente}} el hidrógeno
    \end{itemize>
    
    \vspace{1cm}
    \begin{center}
        \colorbox{azuloscuro!20}{\parbox{0.8\textwidth}{
            \begin{center}
                \LARGE \textbf{El modelo de Bohr fue el primer paso} \\
                \LARGE \textbf{hacia la mecánica cuántica moderna}
            \end{center}
        }}
    \end{center>
\end{frame>

% ==================== SLIDE 29: AGRADECIMIENTOS ====================
\begin{frame}
    \begin{center}
        \Huge \textbf{¡Gracias por su Atención!}
        
        \vspace{2cm}
        \LARGE \textbf{¿Preguntas?}
        
        \vspace{2cm}
        \begin{tikzpicture}
            % Átomo de Bohr decorativo
            \draw[thick] (0,0) circle (1);
            \draw[thick] (0,0) circle (1.5);
            \draw[thick] (0,0) circle (2);
            \fill[red] (0,0) circle (0.15);
            \fill[blue] (1,0) circle (0.1);
            \fill[blue] (-1.5,0) circle (0.1);
            \fill[blue] (0,2) circle (0.1);
        \end{tikzpicture}
        
        \vspace{1cm}
        \Large \textbf{Próxima clase: Dualidad Onda-Partícula}
    \end{center}
\end{frame}

\end{document}