\documentclass{beamer}
\usepackage[utf8]{inputenc}
\usepackage{amsmath}
\usepackage{graphicx}
\usepackage{amssymb} % For symbols like \hbar

\usetheme{AnnArbor}
\usecolortheme{spruce}

\setbeamertemplate{itemize items}[default]
\setbeamertemplate{enumerate items}[default]

% define a darker brown
\definecolor{uwbrown}{HTML}{662200}
% apply dark brown to the item bullet points
\setbeamercolor{item}{fg=uwbrown}

\title[Dualidad Onda-Partícula]{El Enigma Cuántico: Viaje a la Dualidad Onda-Partícula}
\subtitle{Una exploración de cómo la materia y la luz desafían nuestra intuición}
\author{(ruben.velazquez@uteq.edu.mx)}
\institute[UTEQ]{Universidad Tecnológica de Querétaro}
\date{Cuatrimestre Mayo - Agosto 2025}
% \logo{\includegraphics[width=1.5cm]{/Imagenes/Logo_uteq.png}} % Ajusta la ruta y el nombre del archivo del logo

\makeatletter
\setbeamertemplate{footline}
{
	\leavevmode%
	\hbox{%
		\begin{beamercolorbox}[wd=.25\paperwidth,ht=2.25ex,dp=1ex,center]{author in head/foot}%
			\usebeamerfont{author in head/foot}\insertshortauthor
		\end{beamercolorbox}%
		\begin{beamercolorbox}[wd=.55\paperwidth,ht=2.25ex,dp=1ex,center]{title in head/foot}%
			\usebeamerfont{title in head/foot}\insertshorttitle
		\end{beamercolorbox}%
		\begin{beamercolorbox}[wd=.2\paperwidth,ht=2.25ex,dp=1ex,right]{section in head/foot}%
			\usebeamerfont{section in head/foot}\insertframenumber{} / \inserttotalframenumber\hspace*{2ex}
	\end{beamercolorbox}}%
	\vskip0pt%
}
\makeatother

\begin{document}
	
	% Diapositiva 1: Título y Bienvenida
	\begin{frame}
		\titlepage
	\end{frame}
	
	% Diapositiva 2: Hoja de Ruta de la Clase
	\begin{frame}
		\frametitle{¿Qué Exploraremos Hoy?}
		\pause
		Breve introducción al tema: la idea de que las cosas pueden ser ondas y partículas.
		\pause
		\begin{block}{Agenda}
			\begin{enumerate}
				\item Fundamentos Clásicos: ¿Onda o Partícula? \pause
				\item El Nacimiento de lo Cuántico: La Energía en Paquetes. \pause
				\item Evidencia Experimental: Viendo lo Invisible. \pause
				\item Mecánica Cuántica: Las Nuevas Reglas del Juego. \pause
				\item Impacto: Tecnología y Filosofía. \pause
				\item Más Allá: Fronteras Actuales (si el tiempo lo permite).
			\end{enumerate}
		\end{block}
		\pause
		\begin{alertblock}{}
			Mencionar que habrá espacio para preguntas.
		\end{alertblock}
	\end{frame}
	
	% Tabla de Contenidos (Opcional, pero útil)
	\begin{frame}
		\frametitle{Tabla de Contenidos}
		\tableofcontents
	\end{frame}
	
	% Módulo 1: Introducción y Fundamentos Clásicos
	\section{Fundamentos Clásicos}
	
	\begin{frame}
		\frametitle{Visión Clásica: El Dominio de las Partículas}
		\begin{block}{Definición de Partícula (Clásica)}
			\begin{itemize}
				\item Ente localizado.
				\item Masa y posición definidas.
				\item Sigue trayectorias predecibles (Leyes de Newton).
				\item \textit{[Sugerencia Visual: Animación simple de una bola de billar moviéndose y rebotando]}.
			\end{itemize}
		\end{block}
		\pause
		\begin{examples}
			Una canica, un planeta, un grano de arena.
		\end{examples}
	\end{frame}
	
	\begin{frame}
		\frametitle{Visión Clásica: El Reino de las Ondas}
		\begin{block}{Definición de Onda (Clásica)}
			\begin{itemize}
				\item Perturbación que se propaga.
				\item Extensión en el espacio.
				\item Características: Longitud de onda ($\lambda$), frecuencia ($\nu$), amplitud, velocidad.
				\item Luz clásica: onda electromagnética, masa nula.
				\item \textit{[Sugerencia Visual: Animación de ondas en el agua o una cuerda vibrando]}.
			\end{itemize}
		\end{block}
		\pause
		\begin{examples}
			Ondas sonoras, olas en el mar, luz (según la visión clásica predominante post-Maxwell).
		\end{examples}
	\end{frame}
	
	\begin{frame}
		\frametitle{La Luz: ¿Corpuscular u Ondulatoria?}
		Un debate que duró siglos.
		\pause
		\begin{columns}[T] % T option for top alignment
			\begin{column}{0.5\textwidth}
				\begin{block}{Newton (Teoría Corpuscular)}
					\begin{itemize}
						\item Luz = Pequeñas partículas.
						\item Explicaba reflexión y refracción.
						\item \textit{[Sugerencia Visual: Retrato de Newton + diagrama simple de reflexión corpuscular]}.
					\end{itemize}
				\end{block}
			\end{column}
			\begin{column}{0.5\textwidth}
				\begin{block}{Huygens (Teoría Ondulatoria)}
					\begin{itemize}
						\item Luz = Onda.
						\item Explicaba reflexión, refracción, difracción.
						\item \textit{[Sugerencia Visual: Retrato de Huygens + diagrama simple del principio de Huygens]}.
					\end{itemize}
				\end{block}
			\end{column}
		\end{columns}
	\end{frame}
	
	\begin{frame}
		\frametitle{Evidencia a Favor de las Ondas}
		\begin{itemize}
			\item \textbf{Experimento de la Doble Rendija de Young (1801):}
			\begin{itemize}
				\item Demostración crucial del comportamiento ondulatorio de la luz.
				\item Patrones de interferencia (franjas brillantes y oscuras).
				\item \textit{[Sugerencia Visual: Diagrama esquemático del experimento de Young con el patrón de interferencia]}.
			\end{itemize} \pause
			\item \textbf{Trabajo de Fresnel (Difracción):} Base matemática. \pause
			\item \textbf{Teoría Electromagnética de Maxwell (1860s):}
			\begin{itemize}
				\item Luz = Onda electromagnética.
				\item Parecía el fin del debate.
			\end{itemize}
		\end{itemize}
	\end{frame}
	
	\begin{frame}
		\frametitle{Problemas en el Paraíso Ondulatorio}
		A finales del s. XIX, la teoría ondulatoria clásica no podía explicar:
		\begin{enumerate}
			\item \textbf{Radiación del cuerpo negro:} ("Catástrofe ultravioleta"). \pause
			\item \textbf{Efecto fotoeléctrico:} Emisión de electrones por luz. \pause
			\item \textbf{Espectros discretos:} Absorción y emisión atómica.
		\end{enumerate}
		\pause
		\textit{[Sugerencia Visual: Gráfica de la radiación del cuerpo negro mostrando la predicción clásica vs. la observación]}.
		\pause
		\begin{alertblock}{Pregunta para la audiencia}
			¿Qué podría estar fallando?
		\end{alertblock}
	\end{frame}
	
	% Módulo 2: El Nacimiento de la Teoría Cuántica y la Cuantización
	\section{Nacimiento de lo Cuántico}
	
	\begin{frame}
		\frametitle{Max Planck y la Cuantización de la Energía (1900)}
		\begin{itemize}
			\item \textbf{Problema:} Radiación del cuerpo negro. \pause
			\item \textbf{Solución de Planck (Postulado Radical):}
			\begin{itemize}
				\item La energía no se emite/absorbe de forma continua.
				\item Se intercambia en paquetes discretos: "cuantos".
			\end{itemize} \pause
			\item \textbf{Relación Fundamental de Planck:}
			\[ E = h\nu \]
			\begin{itemize}
				\item $E$: Energía del cuanto.
				\item $\nu$: Frecuencia de la radiación.
				\item $h$: Constante de Planck (nueva constante universal).
			\end{itemize} \pause
			\item \textit{[Sugerencia Visual: Retrato de Planck + la ecuación $E = h\nu$ destacada]}. \pause
			\item Implicación: La energía a nivel microscópico es "granulada".
		\end{itemize}
	\end{frame}
	
	\begin{frame}
		\frametitle{Einstein Extiende la Idea: Nacen los Fotones (1905)}
		\begin{itemize}
			\item \textbf{Problema:} Efecto fotoeléctrico.
			\begin{itemize}
				\item La energía de los electrones emitidos depende de la frecuencia (color) de la luz, no de su intensidad.
				\item Existencia de una frecuencia umbral.
			\end{itemize} \pause
			\item \textbf{Explicación de Einstein:}
			\begin{itemize}
				\item La luz misma está compuesta por cuantos de energía: \textbf{fotones}.
				\item Un fotón ($E=h\nu$) interactúa con un electrón.
				\item Si $h\nu > \text{Función Trabajo (energía de enlace del electrón)} \Rightarrow \text{electrón emitido}$.
			\end{itemize} \pause
			\item \textit{[Sugerencia Visual: Diagrama del efecto fotoeléctrico: fotones incidiendo en un metal, electrones siendo emitidos. Mostrar cómo varía con la frecuencia y la intensidad]}. \pause
			\item \textbf{Conclusión:} ¡La luz también tiene propiedades de partícula!
		\end{itemize}
	\end{frame}
	
	\begin{frame}
		\frametitle{La Audaz Hipótesis de De Broglie (1924)}
		\begin{itemize}
			\item \textbf{Razonamiento (Simetría):}
			\begin{itemize}
				\item Si las ondas (luz) pueden comportarse como partículas (fotones)...
				\item ...entonces las partículas (electrones, etc.) deberían poder comportarse como ondas.
			\end{itemize} \pause
			\item \textbf{Relación de De Broglie (Ondas de Materia):}
			\[ \lambda = \frac{h}{p} \]
			\begin{itemize}
				\item $\lambda$: Longitud de onda de la partícula.
				\item $h$: Constante de Planck.
				\item $p$: Momento lineal de la partícula ($mv$).
			\end{itemize} \pause
			\item \textit{[Sugerencia Visual: Retrato de De Broglie + la ecuación $\lambda = h/p$ destacada]}. \pause
			\item Implicación: Toda la materia tiene una naturaleza ondulatoria.
			\begin{itemize}
				\item Objetos macroscópicos: $\lambda$ muy pequeña, indetectable.
				\item Partículas subatómicas (electrones): $\lambda$ significativa.
			\end{itemize}
		\end{itemize}
	\end{frame}
	
	\begin{frame}
		\frametitle{Conectando Mundos: Ondas y Partículas}
		Estas ecuaciones son el corazón de la dualidad.
		\pause
		\begin{block}{Para Fotones (Luz)}
			\begin{itemize}
				\item $E = h\nu$ (Energía de partícula, frecuencia de onda)
				\item $p = h/\lambda$ (Momento de partícula, longitud de onda)
			\end{itemize}
		\end{block}
		\pause
		\begin{block}{Para Partículas de Materia (ej. Electrones)}
			\begin{itemize}
				\item $\lambda = h/p$ (Longitud de onda, momento de partícula)
			\end{itemize}
		\end{block}
		\pause
		\begin{alertblock}{Pregunta para la audiencia}
			¿Qué tan "grande" es la longitud de onda de un electrón en un televisor antiguo? ¿Y la tuya al caminar? (Para ilustrar la escala).
		\end{alertblock}
	\end{frame}
	
	% Módulo 3: Evidencia Experimental de la Dualidad
	\section{Evidencia Experimental}
	
	\begin{frame}
		\frametitle{¡Los Electrones se Comportan como Ondas!}
		\begin{itemize}
			\item \textbf{Experimento de Davisson y Germer (1927):}
			\begin{itemize}
				\item Electrones disparados contra un cristal de Níquel.
				\item Observaron patrones de difracción, ¡similares a los de los rayos X (ondas)!
				\item \textit{[Sugerencia Visual: Esquema del experimento de Davisson-Germer y el patrón de difracción obtenido]}.
			\end{itemize} \pause
			\item \textbf{Experimento de G.P. Thomson (independiente, misma época):}
			\begin{itemize}
				\item Difracción de electrones a través de láminas metálicas delgadas.
			\end{itemize} \pause
			\item \textbf{Confirmación:} Los electrones tienen una longitud de onda, y coincide con $\lambda = h/p$.
		\end{itemize}
	\end{frame}
	
	\begin{frame}
		\frametitle{La Doble Rendija: Ahora con Electrones (y Más)}
		\begin{itemize}
			\item Se adapta el experimento de Young para partículas: electrones, neutrones, átomos, ¡incluso moléculas (fullerenos)! \pause
			\item \textbf{Resultado Sorprendente:} ¡Se observa un patrón de interferencia!
			\begin{itemize}
				\item \textit{[Sugerencia Visual: Diagrama animado (conceptual) de electrones pasando uno a uno por la doble rendija y formando gradualmente el patrón de interferencia en la pantalla]}.
			\end{itemize} \pause
			\item \textbf{La Paradoja:}
			\begin{itemize}
				\item El patrón se forma incluso enviando partículas \textbf{UNA POR UNA}.
				\item Cada partícula parece "pasar por ambas rendijas" e interferir consigo misma.
			\end{itemize}
		\end{itemize}
	\end{frame}
	
	\begin{frame}
		\frametitle{Observar Cambia el Resultado}
		\begin{itemize}
			\item \textbf{Si intentamos detectar por cuál rendija pasa la partícula:}
			\begin{itemize}
				\item Colocamos un "detector" en una de las rendijas.
				\item \textbf{¡El patrón de interferencia DESAPARECE!}
				\item Se observa un patrón de dos franjas, como si fueran partículas clásicas.
				\item \textit{[Sugerencia Visual: Modificación del diagrama anterior, mostrando un "detector" en una rendija y el cambio en el patrón final]}.
			\end{itemize} \pause
			\item \textbf{Conclusión Profunda:} El acto de medir afecta el comportamiento del sistema. La partícula se propaga como onda, pero se detecta como corpúsculo.
		\end{itemize}
	\end{frame}
	
	\begin{frame}
		\frametitle{Explorando los Límites (Variaciones Fascinantes)}
		Breve mención de:
		\begin{itemize}
			\item \textbf{Interferencia con antimateria (positrones):} También muestran dualidad. \pause
			\item \textbf{Experimentos de elección retardada (Wheeler):} La "decisión" de medir el camino o la interferencia se toma \textit{después} de que la partícula pasa las rendijas. \pause
			\item \textbf{Borrador cuántico:} Se puede "borrar" la información del camino y restaurar la interferencia.
		\end{itemize}
		\pause
		Estos experimentos profundizan el misterio y la extrañeza cuántica.
	\end{frame}
	
	% Módulo 4: La Dualidad en el Marco de la Mecánica Cuántica
	\section{Marco de la Mecánica Cuántica}
	
	\begin{frame}
		\frametitle{Describiendo lo Indescriptible: La Función de Onda ($\Psi$)}
		\begin{itemize}
			\item En Mecánica Cuántica, una partícula (ej. electrón) se describe por su \textbf{Función de Onda ($\Psi$)}.
			\begin{itemize}
				\item $\Psi(x, y, z, t)$: Función matemática compleja (espacio y tiempo).
			\end{itemize} \pause
			\item \textbf{Ecuación de Schrödinger (1926):}
			\begin{itemize}
				\item Ley fundamental que gobierna la evolución de $\Psi$.
				\item Análogo cuántico a las leyes de Newton.
				\item \textit{[Sugerencia Visual: Ecuación de Schrödinger (forma general) destacada. No es necesario que la entiendan en detalle, solo su importancia]}.
			\end{itemize} \pause
			\item La función de onda es inherentemente ondulatoria (permite superposición, interferencia).
		\end{itemize}
	\end{frame}
	
	\begin{frame}
		\frametitle{¿Qué Significa $\Psi$? La Probabilidad al Mando}
		\begin{itemize}
			\item \textbf{Max Born (1926):}
			\begin{itemize}
				\item $\Psi$ en sí misma no es una onda física.
				\item $|\Psi|^2$ (el cuadrado de su módulo) = \textbf{Densidad de probabilidad} de encontrar la partícula en un punto.
				\item \textit{[Sugerencia Visual: Gráfica de una función de onda simple y su correspondiente $|\Psi|^2$ como una "nube de probabilidad"]}.
			\end{itemize} \pause
			\item \textbf{Implicaciones:}
			\begin{itemize}
				\item No se puede hablar de posición exacta antes de medir.
				\item La partícula existe en un estado de superposición de posibles posiciones.
				\item La Mecánica Cuántica es intrínsecamente probabilística.
			\end{itemize}
		\end{itemize}
	\end{frame}
	
	\begin{frame}
		\frametitle{El Acto de Medir y sus Consecuencias}
		\begin{itemize}
			\item \textbf{Al medir una propiedad (ej. posición):}
			\begin{itemize}
				\item La función de onda "colapsa".
				\item De una distribución de probabilidades, pasa a un estado con un valor definido (el medido).
				\item \textit{[Sugerencia Visual: Animación conceptual: la "nube" de $|\Psi|^2$ se contrae a un punto al simular una "medición"]}.
			\end{itemize} \pause
			\item Este "colapso" es uno de los aspectos más debatidos y menos comprendidos de la MC.
		\end{itemize}
	\end{frame}
	
	\begin{frame}
		\frametitle{Bohr: Ondas y Partículas, Caras de la Misma Moneda}
		\begin{block}{Principio de Complementariedad (Niels Bohr)}
			\begin{itemize}
				\item Los aspectos ondulatorios y corpusculares son \textbf{complementarios}.
				\item Ambos son necesarios para una descripción completa.
				\item \textbf{No pueden observarse simultáneamente} en el mismo experimento.
				\item El tipo de experimento determina qué aspecto se manifiesta.
			\end{itemize}
		\end{block}
		\pause
		\textit{[Sugerencia Visual: Un diagrama tipo "Yin-Yang" con "Onda" y "Partícula" como las dos mitades]}.
	\end{frame}
	
	\begin{frame}
		\frametitle{Heisenberg: Límites Fundamentales al Conocimiento}
		\begin{block}{Relación de Incertidumbre (Werner Heisenberg, 1927)}
			\begin{itemize}
				\item Límite fundamental a la precisión simultánea de pares de propiedades conjugadas.
				\item Ejemplo clave: Posición ($\Delta x$) y Momento lineal ($\Delta p_x$).
				\[ (\Delta x) (\Delta p_x) \geq \frac{\hbar}{2} \quad (\text{donde } \hbar = h/2\pi) \]
				\item \textit{[Sugerencia Visual: Una balanza, donde al aumentar la precisión de "Posición", disminuye la de "Momento", y viceversa]}.
			\end{itemize}
		\end{block}
		\pause
		\begin{alertblock}{Significado}
			\begin{itemize}
				\item Cuanto más preciso es $\Delta x$, menos preciso es $\Delta p_x$ (y viceversa).
				\item Es una característica intrínseca de la naturaleza, no un defecto de los instrumentos.
			\end{itemize}
		\end{alertblock}
	\end{frame}
	
	\begin{frame}
		\frametitle{Energías Discretas y "Direcciones" Cuánticas (Mención)}
		Aplicar la Ec. de Schrödinger a sistemas confinados (ej. átomo de Hidrógeno):
		\begin{itemize}
			\item Conduce a niveles de energía cuantizados (discretos). \pause
			\item Estados descritos por \textbf{Números Cuánticos} (n, l, m, s). \pause
			\item Principio de Exclusión de Pauli.
		\end{itemize}
		\pause
		Fundamental para entender la estructura atómica y la química.
		\pause
		\textit{[Sugerencia Visual: Diagrama simple de niveles de energía del átomo de Hidrógeno]}.
	\end{frame}
	
	% Módulo 5: Implicaciones, Aplicaciones y Conexión con Física Electrónica
	\section{Impacto: Tecnología y Filosofía}
	
	\begin{frame}
		\frametitle{La Cuántica en tus Dispositivos}
		\begin{itemize}
			\item \textbf{Electrones en Sólidos:}
			\begin{itemize}
				\item Comportamiento ondulatorio crucial para entender el transporte de carga.
				\item Teoría de Bandas (metales, semiconductores, aislantes).
			\end{itemize} \pause
			\item \textbf{Interacción Luz-Electrones:}
			\begin{itemize}
				\item Base de fotodetectores, células solares (excitación óptica).
				\item Láseres (emisión estimulada de fotones).
			\end{itemize} \pause
			\item \textbf{Efecto Túnel:}
			\begin{itemize}
				\item Partículas (ondas) atravesando barreras "imposibles".
				\item Clave en memorias Flash, microscopio de efecto túnel (STM).
			\end{itemize} \pause
			\item \textit{[Sugerencia Visual: Diagrama simple de bandas de energía en un semiconductor y el concepto de efecto túnel]}.
		\end{itemize}
	\end{frame}
	
	\begin{frame}
		\frametitle{Tecnologías Nacidas de la Dualidad (Parte 1)}
		\begin{columns}[T]
			\begin{column}{0.5\textwidth}
				\begin{block}{Microscopio de Electrones}
					\begin{itemize}
						\item Usa la pequeña $\lambda$ de los electrones para alta resolución.
						\item Visualizar átomos y moléculas.
						\item \textit{[Sugerencia Visual: Una imagen impresionante obtenida por un microscopio electrónico]}.
					\end{itemize}
				\end{block}
			\end{column}
			\begin{column}{0.5\textwidth}
				\begin{block}{Láseres}
					\begin{itemize}
						\item Luz coherente y potente gracias a la naturaleza cuántica de la luz y su interacción con la materia.
						\item \textit{[Sugerencia Visual: Aplicaciones diversas de los láseres: medicina, industria, comunicaciones]}.
					\end{itemize}
				\end{block}
			\end{column}
		\end{columns}
	\end{frame}
	
	\begin{frame}
		\frametitle{Tecnologías Nacidas de la Dualidad (Parte 2)}
		\begin{columns}[T]
			\begin{column}{0.5\textwidth}
				\begin{block}{Comunicación Cuántica y Criptografía Cuántica (QKD)}
					\begin{itemize}
						\item Usa propiedades cuánticas (dualidad, incertidumbre) para seguridad.
						\item Detección de espionaje.
					\end{itemize}
				\end{block}
			\end{column}
			\begin{column}{0.5\textwidth}
				\begin{block}{Computación Cuántica (Mención)}
					\begin{itemize}
						\item Aprovecha superposición y entrelazamiento para cálculos revolucionarios.
					\end{itemize}
				\end{block}
			\end{column}
		\end{columns}
		\textit{[Sugerencia Visual: Iconos representando seguridad informática y un chip cuántico conceptual]}.
	\end{frame}
	
	\begin{frame}
		\frametitle{Repensando la Realidad (Implicaciones Filosóficas)}
		\begin{itemize}
			\item La dualidad onda-partícula es \textbf{contraintuitiva}. \pause
			\item \textbf{Cuestiona:}
			\begin{itemize}
				\item La naturaleza de la realidad: ¿Qué "es" un electrón antes de medirlo?
				\item El papel de la medición y el observador.
				\item El determinismo vs. la probabilidad.
			\end{itemize} \pause
			\item \begin{quote}
				"Si la mecánica cuántica no te ha sorprendido profundamente, no la has entendido todavía." - Niels Bohr.
			\end{quote} \pause
			\item \begin{alertblock}{Pregunta para la audiencia}
				¿Cómo cambia esto nuestra visión del mundo?
			\end{alertblock}
		\end{itemize}
	\end{frame}
	
	% Módulo 6: Temas Avanzados y Direcciones Actuales
	\section{Más Allá: Fronteras Actuales}
	
	\begin{frame}
		\frametitle{El Viaje Continúa: Preguntas Abiertas}
		(Opcional, si hay tiempo)
		\begin{itemize}
			\item \textbf{Experimentos con objetos más grandes:}
			\begin{itemize}
				\item ¿Dónde está el límite entre lo cuántico y lo clásico? (Interferencia con moléculas complejas).
			\end{itemize} \pause
			\item \textbf{El Problema de la Medida:}
			\begin{itemize}
				\item ¿Qué es exactamente una "medición"? ¿Cómo ocurre el "colapso"?
				\item Diferentes interpretaciones de la Mecánica Cuántica.
			\end{itemize} \pause
			\item \textbf{Unificación con la Relatividad General.} \pause
			\item \textit{[Sugerencia Visual: Una imagen conceptual de la frontera cuántico-clásica o un diagrama de las diferentes interpretaciones de la MC]}.
		\end{itemize}
	\end{frame}
	
	% Conclusión
	\section{Conclusión}
	
	\begin{frame}
		\frametitle{Dualidad Onda-Partícula: Un Pilar de la Física Moderna}
		\begin{block}{Recapitulación de ideas clave}
			\begin{itemize}
				\item La luz y la materia exhiben propiedades tanto de onda como de partícula.
				\item La descripción depende del experimento.
				\item La Mecánica Cuántica provee el marco formal.
				\item $E=h\nu$, $\lambda=h/p$ son relaciones fundamentales.
				\item Profundas implicaciones tecnológicas y filosóficas.
			\end{itemize}
		\end{block}
		\pause
		La dualidad nos muestra un universo más rico y extraño de lo que imaginábamos.
	\end{frame}
	
	\begin{frame}
		\frametitle{¿Preguntas?}
		\begin{center}
			\Huge \textbf{¿Preguntas?}
		\end{center}
		\vfill
		Abrir el espacio para preguntas de los estudiantes.
		\newline
		Fomentar la discusión.
	\end{frame}
	
	\begin{frame}
		\frametitle{¡Gracias!}
		(Opcional)
		\begin{center}
			\Huge \textbf{¡Gracias por su atención!}
		\end{center}
		\vfill
		Sugerencias de lectura adicional o recursos online (si aplica).
		\newline
		Información de contacto: ruben.velazquez@uteq.edu.mx
	\end{frame}
	
\end{document}
