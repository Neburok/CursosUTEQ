\documentclass[12pt,aspectratio=169]{beamer}
% Paquetes necesarios
\usepackage[utf8]{inputenc}
\usepackage[spanish]{babel}
\usepackage{amsmath,amssymb,amsfonts}
\usepackage{graphicx}
\usepackage{tikz}
\usepackage{pgfplots}
\usepackage{xcolor}
\usepackage{multicol}
\usepackage{textcomp}
\usepackage{siunitx}

% Tema y configuración
\usetheme{Madrid}
\usecolortheme{seahorse}
\usefonttheme{professionalfonts}


% Configuración de TikZ
\usetikzlibrary{shapes,arrows,positioning,decorations.pathmorphing}
\pgfplotsset{compat=1.17}

% Colores personalizados
\definecolor{azuloscuro}{RGB}{25,25,112}
\definecolor{rojoclaro}{RGB}{220,20,60}
\definecolor{verdeclaro}{RGB}{50,205,50}
\definecolor{naranjaclaro}{RGB}{255,140,0}

% Configuración de fuentes más grandes
\setbeamerfont{title}{size=\huge,series=\bfseries}
\setbeamerfont{subtitle}{size=\Large}
\setbeamerfont{frametitle}{size=\Large,series=\bfseries}
\setbeamerfont{framesubtitle}{size=\large}
\setbeamerfont{normal text}{size=\large}
\setbeamerfont{caption}{size=\large}

% Configuración de ecuaciones grandes
\everymath{\displaystyle}
\setbeamertemplate{itemize items}[default]
\setbeamertemplate{enumerate items}[default]

% define a darker brown
\definecolor{uwbrown}{HTML}{662200}
% apply dark brown to the item bullet points
\setbeamercolor{item}{fg=uwbrown}

\title[Espectros Atómicos]{Los Espectros Atómicos}
\subtitle{Fundamentos de la Física Cuántica}
\author{ruben.velazquez@uteq.edu.mx}
\institute[UTEQ]{Universidad Tecnológica de Querétaro}
\date{Cuatrimestre Mayo - Agosto 2025}
\logo{\includegraphics[width=2cm]{../Imagenes/Logo_uteq.png}}

\begin{document}
% ==================== SLIDE 1: TÍTULO ====================

\frame{\titlepage}

% ==================== SLIDE 2: OBJETIVOS ====================
\begin{frame}{Objetivos de Aprendizaje}
    \vspace{1cm}
    \begin{itemize}
        \item[\textbullet] \Large \textbf{Comprender} las limitaciones del modelo clásico
        \vspace{0.5cm}
        \item[\textbullet] \Large \textbf{Analizar} la ecuación empírica de Balmer
        \vspace{0.5cm}
        \item[\textbullet] \Large \textbf{Aplicar} los postulados de Bohr
        \vspace{0.5cm}
        \item[\textbullet] \Large \textbf{Evaluar} la revolución cuántica
    \end{itemize}
\end{frame}
% ==================== SLIDE 3: CONTENIDO ====================
\begin{frame}{Contexto Histórico}
    
    \begin{center}
        \Large \textbf{Los Descubrimientos Clave}
    \end{center}
    
    \begin{center}
        \begin{tikzpicture}[scale=1.2]
            % Línea de tiempo
            \draw[thick,->] (0,0) -- (12,0);
            
            % Marcadores de años
            \foreach \x/\year in {1/1885, 4/1900, 7/1911, 9/1913, 11/1920s} {
                \draw (\x,0) -- (\x,0.3);
                \node[below] at (\x,-0.3) {\Large \textbf{\year}};
            }
            
            % Eventos
            \node[above,text width=2.5cm,align=center] at (1,0.5) {\large \textbf{Balmer} \\ Fórmula empírica};
            \node[above,text width=2.5cm,align=center] at (7,0.5) {\large \textbf{Rutherford} \\ Modelo planetario};
            \node[above,text width=2.5cm,align=center] at (9,0.5) {\large \textbf{Bohr} \\ Postulados cuánticos};
            \node[above,text width=2.5cm,align=center] at (11,0.5) {\large \textbf{Mecánica} \\ \textbf{Cuántica}};
        \end{tikzpicture}
    \end{center}
\end{frame}

% ==================== SLIDE 4: MODELO CLÁSICO ====================
\begin{frame}{El Misterio de las Líneas Espectrales}
    \large \textbf{¿Por qué líneas discretas?}
     
    \begin{center}
        \begin{tikzpicture}[scale=1.5]
            % Espectro continuo
            \fill[left color=red, right color=violet] (0,2) rectangle (8,2.8);
            \node[left] at (0,2.4) {\Large \textbf{Continuo}};
            
            % Espectro de líneas
            \fill[black] (0,0) rectangle (8,0.8);
            \foreach \x/\color in {1.5/red, 3/cyan, 4.5/blue, 6/violet} {
                \fill[\color] (\x,0) rectangle (\x+0.2,0.8);
            }
            \node[left] at (0,0.4) {\Large \textbf{Hidrógeno}};
            
            % Etiquetas
            \node[below] at (1.6,-0.3) {\large H$_\alpha$};
            \node[below] at (3.1,-0.3) {\large H$_\beta$};
            \node[below] at (4.6,-0.3) {\large H$_\gamma$};
            \node[below] at (6.1,-0.3) {\large H$_\delta$};
        \end{tikzpicture}
    \end{center}
       
\end{frame}

% ==================== SLIDE 5: ECUACIÓN DE BALMER ====================
\begin{frame}{La Ecuación Empírica de Balmer}
    \begin{center}
        \large \textbf{El Patrón Matemático (1885)}
    \end{center}
    
    \begin{center}
        \colorbox{azuloscuro!10}{\parbox{0.8\textwidth}{
            \begin{center}
                \Huge \textcolor{azuloscuro}{
                    $$\frac{1}{\lambda} = R_H\left(\frac{1}{n_1^2} - \frac{1}{n_2^2}\right)$$
                }
            \end{center}
        }}
    \end{center}
        
\end{frame}

\begin{frame}
\vspace{1cm}
    \begin{center}
        \begin{tabular}{cl}
            \Large $\lambda$ & \Large Longitud de onda \\[0.3cm]
            \Large $R_H$ & \Large Constante de Rydberg = $1.097 \times 10^7$ m$^{-1}$ \\[0.3cm]
            \Large $n_1, n_2$ & \Large Números enteros ($n_2 > n_1$)
        \end{tabular}
    \end{center}
\end{frame}

% ==================== SLIDE 6: SERIES ESPECTRALES ====================
\begin{frame}{Series Espectrales del Hidrógeno}
    
    \begin{center}
        \Large \textbf{Clasificación por $n_1$}
    \end{center}
    
    \begin{center}
        \begin{tikzpicture}[scale=1.3]
            % Diagrama de niveles
            \foreach \n/\y/\energy in {1/1/-13.6, 2/2.5/-3.4, 3/3.5/-1.51, 4/4.2/-0.85, 5/4.7/0} {
                \draw[thick] (0,\y) -- (8,\y);
                \node[left] at (0,\y) {\Large $n=\n$};
                \node[right] at (8,\y) {\Large \energy eV};
            }
            
            % Transiciones de Lyman (n1=1)
            \draw[thick,red,->] (1.5,2.5) -- (1.5,1.2);
            \draw[thick,red,->] (2,3.5) -- (2,1.2);
            \node[below] at (1.75,0.8) {\large \textcolor{red}{\textbf{Lyman (UV)}}};
            
            % Transiciones de Balmer (n1=2)
            \draw[thick,blue,->] (4.5,3.5) -- (4.5,2.7);
            \draw[thick,blue,->] (5,4.2) -- (5,2.7);
            \node[below] at (4.75,2.5) {\large \textcolor{blue}{\textbf{Balmer (Vis)}}};
            
            % Transiciones de Paschen (n1=3)
            \draw[thick,green,->] (6.5,4.2) -- (6.5,3.7);
            \draw[thick,green,->] (7,4.7) -- (7,3.7);
            \node[below] at (6.75,3.5) {\large \textcolor{green}{\textbf{Paschen (IR)}}};
        \end{tikzpicture}
    \end{center}
\end{frame}

\end{document}