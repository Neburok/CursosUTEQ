\documentclass[12pt,a4paper]{article}
\usepackage[utf8]{inputenc}
\usepackage[spanish]{babel}
\usepackage{amsmath}
\usepackage{amsfonts}
\usepackage{amssymb}
\usepackage{graphicx}
\usepackage{geometry}
\usepackage{hyperref}
\usepackage{array}
\usepackage{multirow}
\usepackage{booktabs}
\usepackage{fancyhdr}
\usepackage{xcolor}
\usepackage{tcolorbox}

% Configuración de página
\geometry{margin=2.5cm}
\pagestyle{fancy}
\fancyhf{}
\fancyhead[L]{Física Moderna - Ingeniería en Nanotecnología}
\fancyhead[R]{Universidad Tecnológica de Querétaro}
\fancyfoot[R]{\thepage}
\fancyfoot[C]{\includegraphics[width=2cm]{../../Imagenes/Logo_uteq}}
% Configuración de colores
\definecolor{uteqblue}{RGB}{0,51,102}
\definecolor{uteqgray}{RGB}{128,128,128}

% Configuración de cajas de texto
\newtcolorbox{objetivobox}{
	colback=blue!5!white,
	colframe=uteqblue,
	title=Objetivos,
	fonttitle=\bfseries
}

\newtcolorbox{instruccionbox}{
	colback=gray!5!white,
	colframe=uteqgray,
	title=Instrucciones,
	fonttitle=\bfseries
}

\newtcolorbox{preguntabox}{
	colback=orange!5!white,
	colframe=orange!50!black,
	title=Preguntas de Análisis,
	fonttitle=\bfseries
}

\begin{document}
	
	% Título principal
	\begin{center}
		{\Huge \textbf{PRÁCTICA 2}}\\[0.3cm]
		{\LARGE \textbf{EFECTO FOTOELÉCTRICO}}\\[0.2cm]
		{\large Física Moderna | Ingeniería en Nanotecnología}\\[1cm]
		\includegraphics[width=5cm]{../../Imagenes/Logo_uteq}
	\end{center}
	
	\vspace{0.5cm}
	
	% Datos generales
	\section*{DATOS GENERALES}
	
	\begin{itemize}
		\item \textbf{Duración:} 45 minutos
		\item \textbf{Modalidad:} Trabajo colaborativo en equipos de 3-4 estudiantes
		\item \textbf{Materiales:} Computadora/tablet con acceso a internet, calculadora, hoja de trabajo impresa
	\end{itemize}

	\vspace{0.5cm}
	
	% Objetivos
	\begin{objetivobox}
		Al finalizar esta actividad, el estudiante será capaz de:
		\begin{itemize}
			\item Comprobar experimentalmente las relaciones del efecto fotoeléctrico mediante una simulación
			\item Determinar la constante de Planck y la función trabajo de diferentes materiales
			\item Analizar gráficamente la relación entre frecuencia y energía cinética de los fotoelectrones
			\item Contrastar las predicciones clásicas con los resultados cuánticos
		\end{itemize}
	\end{objetivobox}
	
	\vspace{1cm}
	\section*{INTRODUCCIÓN}
	
	El simulador PhET ``Efecto Fotoeléctrico'' permite explorar de manera virtual todas las características del efecto fotoeléctrico. En esta actividad, utilizaremos este simulador para verificar experimentalmente la ecuación de Einstein, determinar constantes físicas fundamentales y analizar el comportamiento de diferentes materiales.
	
	% Instrucciones generales
	\begin{instruccionbox}
		\begin{enumerate}
			\item Formen equipos de 3-4 integrantes
			\item Accedan al simulador PhET: \url{https://phet.colorado.edu/es/simulation/photoelectric}
			\item Completen cada una de las secciones siguientes, registrando sus observaciones y resultados
			\item Discutan las preguntas de análisis en equipo
			\item Preparen una breve presentación (2-3 minutos) con sus conclusiones principales
		\end{enumerate}
	\end{instruccionbox}
	
	\newpage
	
	\section{PARTE 1: VERIFICACIÓN DE LA ECUACIÓN DE EINSTEIN}
	
	\subsection{Procedimiento:}
	\begin{enumerate}
		\item Seleccionen el material ``Sodio'' en el simulador
		\item Configuren la intensidad al 50\% y el voltaje de la batería en 0 V
		\item Para cada longitud de onda de la tabla, registren la corriente observada y la energía cinética máxima de los electrones (usando el voltaje de frenado)
		\item Si no hay emisión de electrones para alguna longitud de onda, marcar con ``N/A''
	\end{enumerate}
	
	\subsection{Tabla de datos:}
	
	\begin{center}
		\begin{tabular}{|c|c|c|c|}
			\hline
			\textbf{Longitud de onda (nm)} & \textbf{Frecuencia ($10^{14}$ Hz)} & \textbf{Corriente (nA)} & \textbf{Energía cinética máxima (eV)} \\
			\hline
			400 & & & \\
			\hline
			450 & & & \\
			\hline
			500 & & & \\
			\hline
			550 & & & \\
			\hline
			600 & & & \\
			\hline
			650 & & & \\
			\hline
		\end{tabular}
	\end{center}
	
	\subsection{Análisis:}
	\begin{enumerate}
		\item Calculen la frecuencia correspondiente a cada longitud de onda usando $f = c/\lambda$
		\item Representen gráficamente la energía cinética máxima frente a la frecuencia
		\item Encuentren la ecuación de la recta de mejor ajuste
		\item A partir de la pendiente, determinen el valor experimental de la constante de Planck
		\item A partir del punto de corte con el eje X, determinen la frecuencia umbral y la función trabajo del sodio
	\end{enumerate}
	
	\begin{preguntabox}
		\textbf{Preguntas:}
		\begin{enumerate}
			\item ¿Cómo se compara su valor experimental de la constante de Planck con el valor aceptado ($h = 6.626 \times 10^{-34}$ J·s)?
			\item ¿Cómo se compara la función trabajo obtenida con el valor teórico para el sodio (2.46 eV)?
			\item ¿Existe algún caso donde se observe emisión por debajo de la frecuencia umbral?
		\end{enumerate}
	\end{preguntabox}
	
	\newpage
	
	\section{PARTE 2: EFECTO DE LA INTENSIDAD DE LA LUZ}
	
	\subsection{Procedimiento:}
	\begin{enumerate}
		\item Mantengan seleccionado el material ``Sodio''
		\item Fijen la longitud de onda en 400 nm
		\item Varíen la intensidad según los valores de la tabla y registren la corriente y la energía cinética máxima
	\end{enumerate}
	
	\subsection{Tabla de datos:}
	
	\begin{center}
		\begin{tabular}{|c|c|c|}
			\hline
			\textbf{Intensidad (\%)} & \textbf{Corriente (nA)} & \textbf{Energía cinética máxima (eV)} \\
			\hline
			20 & & \\
			\hline
			40 & & \\
			\hline
			60 & & \\
			\hline
			80 & & \\
			\hline
			100 & & \\
			\hline
		\end{tabular}
	\end{center}
	
	\subsection{Análisis:}
	\begin{enumerate}
		\item Representen gráficamente la corriente frente a la intensidad
		\item Representen gráficamente la energía cinética máxima frente a la intensidad
	\end{enumerate}
	
	\begin{preguntabox}
		\textbf{Preguntas:}
		\begin{enumerate}
			\item ¿Cómo afecta la intensidad de la luz a la corriente fotoeléctrica?
			\item ¿Cómo afecta la intensidad de la luz a la energía cinética máxima de los fotoelectrones?
			\item ¿Estos resultados son coherentes con la teoría clásica o con la teoría cuántica? Expliquen.
		\end{enumerate}
	\end{preguntabox}
	
	\section{PARTE 3: COMPARACIÓN DE DIFERENTES MATERIALES}
	
	\subsection{Procedimiento:}
	\begin{enumerate}
		\item Fijen la longitud de onda en 500 nm y la intensidad al 50\%
		\item Para cada material disponible en el simulador, determinen si hay emisión fotoeléctrica
		\item Si hay emisión, registren la energía cinética máxima usando el voltaje de frenado
	\end{enumerate}
	
	\subsection{Tabla de datos:}
	
	\begin{center}
		\begin{tabular}{|c|c|c|c|}
			\hline
			\textbf{Material} & \textbf{¿Hay emisión? (Sí/No)} & \textbf{Energía cinética máxima (eV)} & \textbf{Función trabajo estimada (eV)} \\
			\hline
			Sodio & & & \\
			\hline
			Zinc & & & \\
			\hline
			Cobre & & & \\
			\hline
			Platino & & & \\
			\hline
			Otro & & & \\
			\hline
		\end{tabular}
	\end{center}
	
	\subsection{Análisis:}
	\begin{enumerate}
		\item Para cada material donde hay emisión, calculen la función trabajo experimental
		\item Ordenen los materiales según su función trabajo, de menor a mayor
	\end{enumerate}
	
	\begin{preguntabox}
		\textbf{Preguntas:}
		\begin{enumerate}
			\item ¿Qué relación existe entre la función trabajo y la facilidad con que un material emite fotoelectrones?
			\item ¿Por qué algunos materiales no presentan efecto fotoeléctrico con la longitud de onda utilizada?
			\item Si quisieran diseñar un detector fotoeléctrico muy sensible a luz visible, ¿qué material elegirían y por qué?
		\end{enumerate}
	\end{preguntabox}
	
	\newpage
	
	\section{PARTE 4: APLICACIÓN A UN PROBLEMA PRÁCTICO}
	
	\subsection{Problema:}
	Se está diseñando un sensor fotoeléctrico para un sistema de seguridad que debe activarse únicamente con luz ultravioleta ($\lambda < 400$ nm) y no con luz visible.
	
	\begin{preguntabox}
		\textbf{Preguntas:}
		\begin{enumerate}
			\item ¿Qué material entre los disponibles sería más adecuado para este sensor? Justifiquen su respuesta.
			\item Calculen la frecuencia de corte y la longitud de onda de corte para el material seleccionado.
			\item Si el sensor se expone accidentalmente a luz visible intensa (por ejemplo, un flash fotográfico), ¿se activaría?
			\item Diseñen un experimento con el simulador para verificar su respuesta al punto 3.
		\end{enumerate}
	\end{preguntabox}
	
	\section{PARTE 5: INTERPRETACIÓN FÍSICA}
	
	\begin{preguntabox}
		\textbf{Preguntas de reflexión:}
		\begin{enumerate}
			\item Expliquen cómo el efecto fotoeléctrico evidencia la naturaleza corpuscular de la luz.
			\item ¿Por qué la emisión fotoeléctrica es instantánea, incluso a intensidades muy bajas?
			\item ¿Cómo se relaciona la ecuación de Einstein ($K_{\text{máx}} = hf - \phi$) con el principio de conservación de la energía?
			\item ¿Qué implicación tiene el efecto fotoeléctrico para el debate histórico sobre si la luz es onda o partícula?
		\end{enumerate}
	\end{preguntabox}
	
	\section*{CONCLUSIONES FINALES}
	
	Basándose en sus resultados y análisis, elaboren un párrafo de conclusiones que resuma:
	\begin{enumerate}
		\item Las principales características del efecto fotoeléctrico verificadas experimentalmente
		\item La validez de la ecuación de Einstein para explicar el fenómeno
		\item Las implicaciones conceptuales para la física cuántica
		\item Posibles aplicaciones tecnológicas basadas en este fenómeno
	\end{enumerate}
	
	\vspace{3cm}
	\hrule
	\vspace{0.3cm}
	
	\section*{ENTREGABLES}
	
	Al finalizar la actividad, cada equipo debe entregar:
	\begin{enumerate}
		\item Esta hoja de trabajo completada con todos los datos y respuestas
		\item Gráficas elaboradas (pueden ser hechas a mano o en computadora)
		\item Un breve reporte con sus conclusiones (máximo una página)
	\end{enumerate}
	
	\section*{CRITERIOS DE EVALUACIÓN}
	
	\begin{center}
		\begin{tabular}{|l|c|}
			\hline
			\textbf{Criterio} & \textbf{Puntuación máxima} \\
			\hline
			Recolección precisa de datos & 20 puntos \\
			\hline
			Cálculos y análisis correctos & 25 puntos \\
			\hline
			Gráficas bien elaboradas & 15 puntos \\
			\hline
			Respuestas a preguntas de análisis & 25 puntos \\
			\hline
			Conclusiones y reflexiones & 15 puntos \\
			\hline
			\textbf{Total} & \textbf{100 puntos} \\
			\hline
		\end{tabular}
	\end{center}
	
	\section*{RECURSOS DE APOYO}
	
	\begin{itemize}
		\item Simulador PhET: \url{https://phet.colorado.edu/es/simulation/photoelectric}
		\item Valores de referencia para funciones trabajo: [tabla proporcionada en clase]
		\item Constantes físicas:
		\begin{itemize}
			\item Constante de Planck: $h = 6.626 \times 10^{-34}$ J·s $= 4.136 \times 10^{-15}$ eV·s
			\item Velocidad de la luz: $c = 3.00 \times 10^{8}$ m/s
			\item Carga del electrón: $e = 1.602 \times 10^{-19}$ C
		\end{itemize}
		\item Conversión útil: $hc \approx 1240$ eV·nm
	\end{itemize}
	
	\vfill
	
	\begin{center}
		\textcolor{uteqgray}{\textit{Universidad Tecnológica de Querétaro | Física Moderna | Cuatrimestre Mayo-Agosto 2025}}
	\end{center}
	
\end{document}