% ============================================================================
% PLANTILLA ESTÁNDAR PARA PRÁCTICAS DE LABORATORIO
% Manual de Física Moderna - Ingeniería en Nanotecnología
% Universidad Tecnológica de Querétaro
% ============================================================================

\documentclass[12pt,a4paper]{article}
\usepackage[utf8]{inputenc}
\usepackage[spanish]{babel}
\usepackage{amsmath}
\usepackage{amsfonts}
\usepackage{amssymb}
\usepackage{graphicx}
\usepackage{geometry}
\usepackage{hyperref}
\usepackage{array}
\usepackage{multirow}
\usepackage{booktabs}
\usepackage{fancyhdr}
\usepackage{xcolor}
\usepackage{tcolorbox}
\usepackage{enumitem}
\usepackage{multicol}
\usepackage{tikz}

% ============================================================================
% CONFIGURACIÓN DE PÁGINA Y ESTILO
% ============================================================================

% Configuración de márgenes
\geometry{margin=2.5cm, top=3cm, bottom=3cm}

% Configuración de colores institucionales
\definecolor{uteqblue}{RGB}{0,51,102}
\definecolor{uteqgray}{RGB}{128,128,128}
\definecolor{uteqgold}{RGB}{255,204,0}
\definecolor{practicegreen}{RGB}{46,125,50}
\definecolor{practiceorange}{RGB}{255,152,0}
\definecolor{practicered}{RGB}{211,47,47}

% Configuración de headers y footers
\pagestyle{fancy}
\fancyhf{}
\fancyhead[L]{\small\textcolor{uteqgray}{Física Moderna - Ingeniería en Nanotecnología}}
\fancyhead[R]{\small\textcolor{uteqgray}{Universidad Tecnológica de Querétaro}}
\fancyfoot[L]{\small\textcolor{uteqgray}{\practicetitle}}
\fancyfoot[C]{\thepage}
\fancyfoot[R]{\small\textcolor{uteqgray}{\practicedate}}
\renewcommand{\headrulewidth}{0.4pt}
\renewcommand{\footrulewidth}{0.4pt}

% ============================================================================
% CONFIGURACIÓN DE CAJAS Y ENTORNOS
% ============================================================================

% Caja para datos generales
\newtcolorbox{datosgeneralesbox}{
	colback=uteqblue!10!white,
	colframe=uteqblue,
	title=\textbf{DATOS GENERALES DE LA PRÁCTICA},
	fonttitle=\bfseries\color{white},
	coltitle=white,
	colbacktitle=uteqblue,
	boxrule=1pt,
	arc=3pt
}

% Caja para objetivos
\newtcolorbox{objetivobox}{
	colback=practicegreen!5!white,
	colframe=practicegreen,
	title=\textbf{OBJETIVOS DE APRENDIZAJE},
	fonttitle=\bfseries\color{white},
	coltitle=white,
	colbacktitle=practicegreen,
	boxrule=1pt,
	arc=3pt
}

% Caja para competencias
\newtcolorbox{competenciabox}{
	colback=uteqgold!20!white,
	colframe=uteqgold!80!black,
	title=\textbf{COMPETENCIAS A DESARROLLAR},
	fonttitle=\bfseries,
	boxrule=1pt,
	arc=3pt
}

% Caja para materiales y recursos
\newtcolorbox{materialesbox}{
	colback=gray!10!white,
	colframe=uteqgray,
	title=\textbf{MATERIALES Y RECURSOS},
	fonttitle=\bfseries\color{white},
	coltitle=white,
	colbacktitle=uteqgray,
	boxrule=1pt,
	arc=3pt
}

% Caja para marco teórico
\newtcolorbox{teoriabox}{
	colback=blue!5!white,
	colframe=blue!50!black,
	title=\textbf{MARCO TEÓRICO},
	fonttitle=\bfseries,
	boxrule=1pt,
	arc=3pt
}

% Caja para instrucciones
\newtcolorbox{instruccionbox}{
	colback=practiceorange!10!white,
	colframe=practiceorange,
	title=\textbf{INSTRUCCIONES},
	fonttitle=\bfseries\color{white},
	coltitle=white,
	colbacktitle=practiceorange,
	boxrule=1pt,
	arc=3pt
}

% Caja para preguntas de análisis
\newtcolorbox{preguntabox}{
	colback=purple!5!white,
	colframe=purple!50!black,
	title=\textbf{PREGUNTAS DE ANÁLISIS},
	fonttitle=\bfseries,
	boxrule=1pt,
	arc=3pt
}

% Caja para nota importante
\newtcolorbox{notabox}{
	colback=practicered!10!white,
	colframe=practicered,
	title=\textbf{NOTA IMPORTANTE},
	fonttitle=\bfseries\color{white},
	coltitle=white,
	colbacktitle=practicered,
	boxrule=1pt,
	arc=3pt
}

% Caja para evaluación
\newtcolorbox{evaluacionbox}{
	colback=teal!10!white,
	colframe=teal!70!black,
	title=\textbf{EVALUACIÓN Y CRITERIOS},
	fonttitle=\bfseries,
	boxrule=1pt,
	arc=3pt
}

% Caja para conclusiones
\newtcolorbox{conclusionbox}{
	colback=green!5!white,
	colframe=green!60!black,
	title=\textbf{CONCLUSIONES Y REFLEXIÓN},
	fonttitle=\bfseries,
	boxrule=1pt,
	arc=3pt
}

% ============================================================================
% CONFIGURACIÓN DE HYPERLINKS
% ============================================================================

\hypersetup{
	colorlinks=true,
	linkcolor=uteqblue,
	filecolor=magenta,      
	urlcolor=cyan,
	pdftitle={Práctica de Laboratorio - Física Moderna},
	pdfauthor={Universidad Tecnológica de Querétaro},
}

% ============================================================================
% COMANDOS PERSONALIZADOS PARA INFORMACIÓN DE LA PRÁCTICA
% ============================================================================

% Comandos que deben ser definidos al inicio de cada práctica
\newcommand{\practicenumber}{XX} % Número de la práctica
\newcommand{\practicetitle}{TÍTULO DE LA PRÁCTICA} % Título de la práctica
\newcommand{\practiceunit}{Unidad X} % Unidad temática
\newcommand{\practicesubtopic}{Subtema específico} % Subtema
\newcommand{\practiceduration}{XX minutos} % Duración
\newcommand{\practicemodality}{Presencial asistida por tecnología} % Modalidad
\newcommand{\practicedate}{Cuatrimestre Mayo-Agosto 2025} % Fecha
\newcommand{\practicesimulator}{Nombre del simulador} % Simulador principal

% ============================================================================
% INICIO DEL DOCUMENTO
% ============================================================================

\begin{document}
	
	% ============================================================================
	% ENCABEZADO DE LA PRÁCTICA
	% ============================================================================
	
	\begin{center}
		% Logo institucional
		\includegraphics[width=3cm]{logo_uteq.png}\\[0.3cm]
		
		% Título principal
		{\LARGE \textcolor{uteqblue}{\textbf{PRÁCTICA No. \practicenumber}}}\\[0.2cm]
		{\Huge \textcolor{uteqblue}{\textbf{\practicetitle}}}\\[0.3cm]
		{\large \textcolor{uteqgray}{\textbf{Manual de Prácticas de Laboratorio}}}\\[0.2cm]
		{\normalsize \textcolor{uteqgray}{Física Moderna | Ingeniería en Nanotecnología}}\\[0.5cm]
	\end{center}
	
	% ============================================================================
	% DATOS GENERALES
	% ============================================================================
	
	\begin{datosgeneralesbox}
		\begin{multicols}{2}
			\textbf{Número de Práctica:} \practicenumber\\
			\textbf{Unidad Temática:} \practiceunit\\
			\textbf{Tema/Subtema:} \practicesubtopic\\
			\textbf{Duración:} \practiceduration\\
			
			\columnbreak
			
			\textbf{Modalidad:} \practicemodality\\
			\textbf{Simulador Principal:} \practicesimulator\\
			\textbf{Tipo de Actividad:} Trabajo colaborativo\\
			\textbf{Tamaño de Equipo:} 3-4 estudiantes\\
		\end{multicols}
	\end{datosgeneralesbox}
	
	% ============================================================================
	% SECCIÓN 1: OBJETIVOS DE APRENDIZAJE
	% ============================================================================
	
	\section{OBJETIVOS DE APRENDIZAJE}
	
	\begin{objetivobox}
		\textbf{Al finalizar esta práctica, el estudiante será capaz de:}
		\begin{enumerate}
			\item [Objetivo específico 1]
			\item [Objetivo específico 2]
			\item [Objetivo específico 3]
			\item [Objetivo específico 4]
			\item [Objetivo específico 5]
		\end{enumerate}
	\end{objetivobox}
	
	% ============================================================================
	% SECCIÓN 2: COMPETENCIAS A DESARROLLAR
	% ============================================================================
	
	\section{COMPETENCIAS A DESARROLLAR}
	
	\begin{competenciabox}
		\begin{multicols}{2}
			\textbf{Competencias Disciplinares:}
			\begin{itemize}[leftmargin=1cm]
				\item [Competencia disciplinar 1]
				\item [Competencia disciplinar 2]
				\item [Competencia disciplinar 3]
			\end{itemize}
			
			\textbf{Competencias Profesionales:}
			\begin{itemize}[leftmargin=1cm]
				\item [Competencia profesional 1]
				\item [Competencia profesional 2]
				\item [Competencia profesional 3]
			\end{itemize}
			
			\columnbreak
			
			\textbf{Competencias Transversales:}
			\begin{itemize}[leftmargin=1cm]
				\item [Competencia transversal 1]
				\item [Competencia transversal 2]
				\item [Competencia transversal 3]
			\end{itemize}
			
			\textbf{Competencias Tecnológicas:}
			\begin{itemize}[leftmargin=1cm]
				\item [Competencia tecnológica 1]
				\item [Competencia tecnológica 2]
				\item [Competencia tecnológica 3]
			\end{itemize}
		\end{multicols}
	\end{competenciabox}
	
	% ============================================================================
	% SECCIÓN 3: MATERIALES Y RECURSOS
	% ============================================================================
	
	\section{MATERIALES Y RECURSOS}
	
	\begin{materialesbox}
		\begin{multicols}{2}
			\textbf{Recursos Tecnológicos:}
			\begin{itemize}[leftmargin=1cm]
				\item Computadora/tablet con navegador actualizado
				\item Conexión a internet estable
				\item [Simulador específico]
				\item [Herramienta adicional]
			\end{itemize}
			
			\textbf{Materiales de Trabajo:}
			\begin{itemize}[leftmargin=1cm]
				\item Calculadora científica
				\item Hoja de trabajo impresa
				\item Material para gráficas
				\item [Material específico]
			\end{itemize}
			
			\columnbreak
			
			\textbf{Recursos Digitales:}
			\begin{itemize}[leftmargin=1cm]
				\item Simulador PhET: [URL específico]
				\item [Herramienta online específica]
				\item [Recurso complementario]
			\end{itemize}
			
			\textbf{Documentos de Apoyo:}
			\begin{itemize}[leftmargin=1cm]
				\item Manual de usuario del simulador
				\item Tablas de constantes físicas
				\item Guía de análisis de datos
				\item [Documento específico]
			\end{itemize}
		\end{multicols}
	\end{materialesbox}
	
	% ============================================================================
	% SECCIÓN 4: MARCO TEÓRICO
	% ============================================================================
	
	\section{MARCO TEÓRICO}
	
	\begin{teoriabox}
		\textbf{Conceptos Fundamentales:}
		
		[Aquí se incluye una explicación concisa de los conceptos teóricos necesarios para la práctica, incluyendo:]
		
		\begin{itemize}
			\item Definiciones clave
			\item Ecuaciones fundamentales
			\item Principios físicos involucrados
			\item Relaciones matemáticas importantes
		\end{itemize}
		
		\textbf{Ecuaciones Principales:}
		
		\begin{align}
			\text{Ecuación 1: } \quad & [Ecuación con descripción] \\
			\text{Ecuación 2: } \quad & [Ecuación con descripción] \\
			\text{Ecuación 3: } \quad & [Ecuación con descripción]
		\end{align}
		
		\textbf{Conexión con Nanotecnología:}
		
		[Explicación de cómo los conceptos se relacionan con aplicaciones en nanotecnología]
	\end{teoriabox}
	
	% ============================================================================
	% SECCIÓN 5: INSTRUCCIONES GENERALES
	% ============================================================================
	
	\section{INSTRUCCIONES GENERALES}
	
	\begin{instruccionbox}
		\textbf{Antes de comenzar:}
		\begin{enumerate}
			\item Formen equipos de 3-4 integrantes
			\item Revisen el marco teórico y los objetivos de la práctica
			\item Verifiquen el acceso al simulador: [URL específico]
			\item Preparen los materiales de trabajo necesarios
			\item Designen roles dentro del equipo (coordinador, secretario, analista, etc.)
		\end{enumerate}
		
		\textbf{Durante la práctica:}
		\begin{enumerate}
			\item Sigan cuidadosamente las instrucciones de cada parte
			\item Registren todas las observaciones y mediciones
			\item Discutan los resultados en equipo antes de continuar
			\item Consulten dudas con el instructor cuando sea necesario
			\item Mantengan un registro ordenado de sus datos
		\end{enumerate}
		
		\textbf{Al finalizar:}
		\begin{enumerate}
			\item Completen todas las secciones de análisis
			\item Elaboren las conclusiones correspondientes
			\item Preparen una breve presentación de sus resultados
			\item Entreguen todos los materiales solicitados
		\end{enumerate}
	\end{instruccionbox}
	
	% ============================================================================
	% SECCIÓN 6: DESARROLLO DE LA PRÁCTICA
	% ============================================================================
	
	\section{DESARROLLO DE LA PRÁCTICA}
	
	% ----------------------------------------------------------------------------
	% PARTE 1
	% ----------------------------------------------------------------------------
	
	\subsection{PARTE 1: [TÍTULO DE LA PRIMERA PARTE]}
	
	\subsubsection{Procedimiento:}
	\begin{enumerate}
		\item [Paso 1 detallado]
		\item [Paso 2 detallado]
		\item [Paso 3 detallado]
		\item [Paso 4 detallado]
		\item [Paso 5 detallado]
	\end{enumerate}
	
	\subsubsection{Tabla de datos:}
	
	\begin{center}
		\begin{tabular}{|c|c|c|c|}
			\hline
			\textbf{Parámetro 1} & \textbf{Parámetro 2} & \textbf{Parámetro 3} & \textbf{Resultado} \\
			\hline
			& & & \\
			\hline
			& & & \\
			\hline
			& & & \\
			\hline
			& & & \\
			\hline
			& & & \\
			\hline
		\end{tabular}
	\end{center}
	
	\subsubsection{Análisis de resultados:}
	\begin{enumerate}
		\item [Instrucción de análisis 1]
		\item [Instrucción de análisis 2]
		\item [Instrucción de análisis 3]
		\item [Instrucción de análisis 4]
	\end{enumerate}
	
	\begin{preguntabox}
		\textbf{Preguntas de Análisis - Parte 1:}
		\begin{enumerate}
			\item [Pregunta conceptual 1]
			
			\vspace{1.5cm}
			
			\item [Pregunta conceptual 2]
			
			\vspace{1.5cm}
			
			\item [Pregunta de aplicación 1]
			
			\vspace{1.5cm}
			
			\item [Pregunta de síntesis 1]
			
			\vspace{1.5cm}
		\end{enumerate}
	\end{preguntabox}
	
	% ----------------------------------------------------------------------------
	% PARTE 2
	% ----------------------------------------------------------------------------
	
	\subsection{PARTE 2: [TÍTULO DE LA SEGUNDA PARTE]}
	
	\subsubsection{Procedimiento:}
	\begin{enumerate}
		\item [Paso 1 detallado]
		\item [Paso 2 detallado]
		\item [Paso 3 detallado]
		\item [Paso 4 detallado]
	\end{enumerate}
	
	\subsubsection{Tabla de datos:}
	
	\begin{center}
		\begin{tabular}{|c|c|c|c|}
			\hline
			\textbf{Variable 1} & \textbf{Variable 2} & \textbf{Observación} & \textbf{Cálculo} \\
			\hline
			& & & \\
			\hline
			& & & \\
			\hline
			& & & \\
			\hline
			& & & \\
			\hline
		\end{tabular}
	\end{center}
	
	\begin{preguntabox}
		\textbf{Preguntas de Análisis - Parte 2:}
		\begin{enumerate}
			\item [Pregunta específica parte 2]
			
			\vspace{1.5cm}
			
			\item [Pregunta de relación]
			
			\vspace{1.5cm}
			
			\item [Pregunta de aplicación tecnológica]
			
			\vspace{1.5cm}
		\end{enumerate}
	\end{preguntabox}
	
	% ----------------------------------------------------------------------------
	% PARTE 3
	% ----------------------------------------------------------------------------
	
	\subsection{PARTE 3: [TÍTULO DE LA TERCERA PARTE]}
	
	[Estructura similar para las partes adicionales que requiera la práctica]
	
	% ============================================================================
	% SECCIÓN 7: APLICACIONES Y CASOS PRÁCTICOS
	% ============================================================================
	
	\section{APLICACIONES Y CASOS PRÁCTICOS}
	
	\subsection{Aplicación en Nanotecnología:}
	
	[Descripción de cómo los conceptos estudiados se aplican específicamente en nanotecnología]
	
	\subsection{Caso de Estudio:}
	
	[Presentación de un caso real donde se utilicen los principios estudiados]
	
	\begin{preguntabox}
		\textbf{Análisis del Caso de Estudio:}
		\begin{enumerate}
			\item ¿Cómo se relacionan los resultados obtenidos con el caso presentado?
			
			\vspace{2cm}
			
			\item ¿Qué aplicaciones adicionales podrían derivarse de estos principios?
			
			\vspace{2cm}
			
			\item ¿Cuáles serían las ventajas y limitaciones de esta aplicación?
			
			\vspace{2cm}
		\end{enumerate}
	\end{preguntabox}
	
	% ============================================================================
	% SECCIÓN 8: EVALUACIÓN FORMATIVA
	% ============================================================================
	
	\section{EVALUACIÓN FORMATIVA}
	
	\begin{evaluacionbox}
		\textbf{Autoevaluación:} Marca con una X tu nivel de comprensión
		
		\begin{center}
			\begin{tabular}{|p{6cm}|c|c|c|c|}
				\hline
				\textbf{Criterio} & \textbf{Excelente} & \textbf{Bueno} & \textbf{Regular} & \textbf{Necesito ayuda} \\
				\hline
				Comprendo los conceptos teóricos fundamentales & & & & \\
				\hline
				Puedo utilizar el simulador correctamente & & & & \\
				\hline
				Analizo correctamente los datos obtenidos & & & & \\
				\hline
				Relaciono la teoría con las aplicaciones prácticas & & & & \\
				\hline
				Trabajo efectivamente en equipo & & & & \\
				\hline
			\end{tabular}
		\end{center}
		
		\textbf{Coevaluación del Equipo:}
		\begin{itemize}
			\item ¿Cómo calificarían el trabajo colaborativo del equipo? \hrulefill
			\item ¿Qué aspectos pueden mejorar para la siguiente práctica? \hrulefill
			\item ¿Todos los miembros participaron activamente? \hrulefill
		\end{itemize}
	\end{evaluacionbox}
	
	% ============================================================================
	% SECCIÓN 9: CONCLUSIONES Y REFLEXIÓN
	% ============================================================================
	
	\section{CONCLUSIONES Y REFLEXIÓN}
	
	\begin{conclusionbox}
		\textbf{Síntesis de Resultados:}
		
		Elaboren un párrafo que resuma los principales hallazgos de la práctica:
		
		\vspace{3cm}
		
		\textbf{Conexión con Objetivos de Aprendizaje:}
		
		¿Cómo esta práctica contribuyó al logro de los objetivos planteados?
		
		\vspace{3cm}
		
		\textbf{Aplicaciones en Ingeniería en Nanotecnología:}
		
		¿Cómo pueden aplicar estos conocimientos en su futura práctica profesional?
		
		\vspace{3cm}
		
		\textbf{Reflexión Personal:}
		
		¿Qué fue lo más interesante o desafiante de esta práctica?
		
		\vspace{3cm}
	\end{conclusionbox}
	
	% ============================================================================
	% SECCIÓN 10: ENTREGABLES Y CRITERIOS DE EVALUACIÓN
	% ============================================================================
	
	\section{ENTREGABLES Y CRITERIOS DE EVALUACIÓN}
	
	\subsection{Entregables:}
	
	\begin{enumerate}
		\item Hoja de trabajo completada con todos los datos y análisis
		\item Gráficas elaboradas (digitales o manuales)
		\item Reporte de conclusiones (máximo 1 página)
		\item Presentación breve de resultados (3-5 minutos)
	\end{enumerate}
	
	\subsection{Criterios de Evaluación:}
	
	\begin{center}
		\begin{tabular}{|l|c|c|}
			\hline
			\textbf{Criterio} & \textbf{Puntuación máxima} & \textbf{Puntuación obtenida} \\
			\hline
			Recolección precisa de datos & 20 puntos & \\
			\hline
			Análisis correcto de resultados & 25 puntos & \\
			\hline
			Calidad de gráficas y presentación & 15 puntos & \\
			\hline
			Respuestas a preguntas de análisis & 25 puntos & \\
			\hline
			Conclusiones y reflexiones & 15 puntos & \\
			\hline
			\textbf{TOTAL} & \textbf{100 puntos} & \\
			\hline
		\end{tabular}
	\end{center}
	
	% ============================================================================
	% SECCIÓN 11: RECURSOS COMPLEMENTARIOS
	% ============================================================================
	
	\section{RECURSOS COMPLEMENTARIOS}
	
	\subsection{Enlaces y Simuladores:}
	\begin{itemize}
		\item Simulador principal: [URL específico]
		\item [Recurso complementario 1]
		\item [Recurso complementario 2]
		\item [Recurso complementario 3]
	\end{itemize}
	
	\subsection{Lecturas Recomendadas:}
	\begin{itemize}
		\item [Referencia bibliográfica 1]
		\item [Referencia bibliográfica 2]
		\item [Referencia bibliográfica 3]
	\end{itemize}
	
	\subsection{Videos Educativos:}
	\begin{itemize}
		\item [Video recomendado 1 con duración]
		\item [Video recomendado 2 con duración]
		\item [Video recomendado 3 con duración]
	\end{itemize}
	
	% ============================================================================
	% SECCIÓN 12: CONSTANTES Y FÓRMULAS DE REFERENCIA
	% ============================================================================
	
	\section{CONSTANTES Y FÓRMULAS DE REFERENCIA}
	
	\begin{notabox}
		\textbf{Constantes Físicas Fundamentales:}
		\begin{multicols}{2}
			\begin{itemize}[leftmargin=1cm]
				\item Constante de Planck: $h = 6.626 \times 10^{-34}$ J·s
				\item Velocidad de la luz: $c = 3.00 \times 10^{8}$ m/s
				\item Carga del electrón: $e = 1.602 \times 10^{-19}$ C
				\item Masa del electrón: $m_e = 9.109 \times 10^{-31}$ kg
				\item [Constante específica de la práctica]
			\end{itemize}
			
			\columnbreak
			
			\begin{itemize}[leftmargin=1cm]
				\item Constante de Boltzmann: $k_B = 1.381 \times 10^{-23}$ J/K
				\item [Conversión útil específica]
				\item [Constante específica 2]
				\item [Constante específica 3]
			\end{itemize}
		\end{multicols}
	\end{notabox}
	
	% ============================================================================
	% PIE DE PÁGINA
	% ============================================================================
	
	\vfill
	
	\begin{center}
		\hrule
		\vspace{0.3cm}
		\textcolor{uteqgray}{\textit{Universidad Tecnológica de Querétaro | Manual de Prácticas de Laboratorio}}\\
		\textcolor{uteqgray}{\textit{Física Moderna | Cuatrimestre Mayo-Agosto 2025}}
	\end{center}
	
\end{document}

% ============================================================================
% INSTRUCCIONES DE USO DE LA PLANTILLA
% ============================================================================

% PARA USAR ESTA PLANTILLA:
% 
% 1. Copiar este archivo y renombrarlo según la práctica específica
% 
% 2. Definir las variables al inicio del documento:
%    \renewcommand{\practicenumber}{01}
%    \renewcommand{\practicetitle}{RADIACIÓN DE CUERPO NEGRO}
%    \renewcommand{\practiceunit}{Unidad I: Fundamentos de Teoría Cuántica}
%    \renewcommand{\practicesubtopic}{Cuantización de la energía}
%    \renewcommand{\practiceduration}{45 minutos}
%    \renewcommand{\practicesimulator}{PhET Blackbody Spectrum}
%
% 3. Completar cada sección con el contenido específico de la práctica
%
% 4. Ajustar el número de partes según sea necesario
%
% 5. Personalizar las tablas de datos según los parámetros específicos
%
% 6. Adaptar las preguntas de análisis al tema específico
%
% 7. Incluir las imágenes necesarias (logo_uteq.png, capturas del simulador, etc.)
%
% 8. Revisar y ajustar los recursos complementarios específicos
%
% Esta plantilla garantiza:
% - Uniformidad visual en todas las prácticas
% - Estructura pedagógica consistente
% - Facilidad de uso y adaptación
% - Profesionalismo en la presentación
% - Cumplimiento de estándares educativos