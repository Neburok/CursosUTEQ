% ============================================================================
% PRÁCTICA 01: RADIACIÓN DE CUERPO NEGRO
% Manual de Física Moderna - Ingeniería en Nanotecnología
% Universidad Tecnológica de Querétaro
% ============================================================================

\documentclass[12pt,a4paper]{article}
\usepackage[utf8]{inputenc}
\usepackage[spanish]{babel}
\usepackage{amsmath}
\usepackage{amsfonts}
\usepackage{amssymb}
\usepackage{graphicx}
\usepackage{geometry}
\usepackage{hyperref}
\usepackage{array}
\usepackage{multirow}
\usepackage{booktabs}
\usepackage{fancyhdr}
\usepackage{xcolor}
\usepackage{tcolorbox}
\usepackage{enumitem}
\usepackage{multicol}
\usepackage{tikz}

% ============================================================================
% CONFIGURACIÓN DE PÁGINA Y ESTILO
% ============================================================================

% Configuración de márgenes
\geometry{margin=2.5cm, top=3cm, bottom=3cm}

% Configuración de colores institucionales
\definecolor{uteqblue}{RGB}{0,51,102}
\definecolor{uteqgray}{RGB}{128,128,128}
\definecolor{uteqgold}{RGB}{255,204,0}
\definecolor{practicegreen}{RGB}{46,125,50}
\definecolor{practiceorange}{RGB}{255,152,0}
\definecolor{practicered}{RGB}{211,47,47}

% Definición de variables específicas de la práctica
\newcommand{\practicenumber}{01}
\newcommand{\practicetitle}{RADIACIÓN DE CUERPO NEGRO}
\newcommand{\practiceunit}{Unidad I: Fundamentos de Teoría Cuántica}
\newcommand{\practicesubtopic}{Cuantización de la energía y Ley de Planck}
\newcommand{\practiceduration}{45 minutos}
\newcommand{\practicemodality}{Presencial asistida por tecnología}
\newcommand{\practicedate}{Cuatrimestre Mayo-Agosto 2025}
\newcommand{\practicesimulator}{PhET Blackbody Spectrum}

% Configuración de headers y footers
\pagestyle{fancy}
\fancyhf{}
\fancyhead[L]{\small\textcolor{uteqgray}{Física Moderna - Ingeniería en Nanotecnología}}
\fancyhead[R]{\small\textcolor{uteqgray}{Universidad Tecnológica de Querétaro}}
\fancyfoot[L]{\small\textcolor{uteqgray}{\practicetitle}}
\fancyfoot[C]{\thepage}
\fancyfoot[R]{\small\textcolor{uteqgray}{\practicedate}}
\renewcommand{\headrulewidth}{0.4pt}
\renewcommand{\footrulewidth}{0.4pt}

% ============================================================================
% CONFIGURACIÓN DE CAJAS Y ENTORNOS
% ============================================================================

% Caja para datos generales
\newtcolorbox{datosgeneralesbox}{
	colback=uteqblue!10!white,
	colframe=uteqblue,
	title=\textbf{DATOS GENERALES DE LA PRÁCTICA},
	fonttitle=\bfseries\color{white},
	coltitle=white,
	colbacktitle=uteqblue,
	boxrule=1pt,
	arc=3pt
}

% Caja para objetivos
\newtcolorbox{objetivobox}{
	colback=practicegreen!5!white,
	colframe=practicegreen,
	title=\textbf{OBJETIVOS DE APRENDIZAJE},
	fonttitle=\bfseries\color{white},
	coltitle=white,
	colbacktitle=practicegreen,
	boxrule=1pt,
	arc=3pt
}

% Caja para competencias
\newtcolorbox{competenciabox}{
	colback=uteqgold!20!white,
	colframe=uteqgold!80!black,
	title=\textbf{COMPETENCIAS A DESARROLLAR},
	fonttitle=\bfseries,
	boxrule=1pt,
	arc=3pt
}

% Caja para materiales y recursos
\newtcolorbox{materialesbox}{
	colback=gray!10!white,
	colframe=uteqgray,
	title=\textbf{MATERIALES Y RECURSOS},
	fonttitle=\bfseries\color{white},
	coltitle=white,
	colbacktitle=uteqgray,
	boxrule=1pt,
	arc=3pt
}

% Caja para marco teórico
\newtcolorbox{teoriabox}{
	colback=blue!5!white,
	colframe=blue!50!black,
	title=\textbf{MARCO TEÓRICO},
	fonttitle=\bfseries,
	boxrule=1pt,
	arc=3pt
}

% Caja para instrucciones
\newtcolorbox{instruccionbox}{
	colback=practiceorange!10!white,
	colframe=practiceorange,
	title=\textbf{INSTRUCCIONES},
	fonttitle=\bfseries\color{white},
	coltitle=white,
	colbacktitle=practiceorange,
	boxrule=1pt,
	arc=3pt
}

% Caja para preguntas de análisis
\newtcolorbox{preguntabox}{
	colback=purple!5!white,
	colframe=purple!50!black,
	title=\textbf{PREGUNTAS DE ANÁLISIS},
	fonttitle=\bfseries,
	boxrule=1pt,
	arc=3pt
}

% Caja para nota importante
\newtcolorbox{notabox}{
	colback=practicered!10!white,
	colframe=practicered,
	title=\textbf{NOTA IMPORTANTE},
	fonttitle=\bfseries\color{white},
	coltitle=white,
	colbacktitle=practicered,
	boxrule=1pt,
	arc=3pt
}

% Caja para evaluación
\newtcolorbox{evaluacionbox}{
	colback=teal!10!white,
	colframe=teal!70!black,
	title=\textbf{EVALUACIÓN Y CRITERIOS},
	fonttitle=\bfseries,
	boxrule=1pt,
	arc=3pt
}

% Caja para conclusiones
\newtcolorbox{conclusionbox}{
	colback=green!5!white,
	colframe=green!60!black,
	title=\textbf{CONCLUSIONES Y REFLEXIÓN},
	fonttitle=\bfseries,
	boxrule=1pt,
	arc=3pt
}

% Configuración de hyperlinks
\hypersetup{
	colorlinks=true,
	linkcolor=uteqblue,
	filecolor=magenta,      
	urlcolor=cyan,
	pdftitle={Práctica 01: Radiación de Cuerpo Negro},
	pdfauthor={Universidad Tecnológica de Querétaro},
}

% ============================================================================
% INICIO DEL DOCUMENTO
% ============================================================================

\begin{document}
	
	% ============================================================================
	% ENCABEZADO DE LA PRÁCTICA
	% ============================================================================
	
	\begin{center}
		% Logo institucional
		\includegraphics[width=3cm]{../../Imagenes/Logo_uteq}\\[0.3cm]
		
		% Título principal
		{\LARGE \textcolor{uteqblue}{\textbf{PRÁCTICA No. \practicenumber}}}\\[0.2cm]
		{\Huge \textcolor{uteqblue}{\textbf{\practicetitle}}}\\[0.3cm]
		{\large \textcolor{uteqgray}{\textbf{Manual de Prácticas de Laboratorio}}}\\[0.2cm]
		{\normalsize \textcolor{uteqgray}{Física Moderna | Ingeniería en Nanotecnología}}\\[0.5cm]
	\end{center}
	
	% ============================================================================
	% DATOS GENERALES
	% ============================================================================
	
	\begin{datosgeneralesbox}
		\begin{multicols}{2}
			\textbf{Número de Práctica:} \practicenumber\\
			\textbf{Unidad Temática:} \practiceunit\\
			\textbf{Tema/Subtema:} \practicesubtopic\\
			\textbf{Duración:} \practiceduration\\
			
			\columnbreak
			
			\textbf{Modalidad:} \practicemodality\\
			\textbf{Simulador Principal:} \practicesimulator\\
			\textbf{Tipo de Actividad:} Trabajo colaborativo\\
			\textbf{Tamaño de Equipo:} 3-4 estudiantes\\
		\end{multicols}
	\end{datosgeneralesbox}
	
	% ============================================================================
	% SECCIÓN 1: OBJETIVOS DE APRENDIZAJE
	% ============================================================================
	
	\section{OBJETIVOS DE APRENDIZAJE}
	
	\begin{objetivobox}
		\textbf{Al finalizar esta práctica, el estudiante será capaz de:}
		\begin{enumerate}
			\item Analizar experimentalmente la distribución espectral de la radiación de cuerpo negro a diferentes temperaturas
			\item Verificar las leyes de Wien y Stefan-Boltzmann utilizando datos de simulación
			\item Comparar las predicciones de la física clásica (Rayleigh-Jeans) con el modelo cuántico de Planck
			\item Interpretar la ``catástrofe ultravioleta'' y comprender cómo la cuantización resolvió esta crisis
			\item Relacionar los principios de radiación de cuerpo negro con aplicaciones en nanotecnología y astrofísica
		\end{enumerate}
	\end{objetivobox}
	
	% ============================================================================
	% SECCIÓN 2: COMPETENCIAS A DESARROLLAR
	% ============================================================================
	
	\section{COMPETENCIAS A DESARROLLAR}
	
	\begin{competenciabox}
		\begin{multicols}{2}
			\textbf{Competencias Disciplinares:}
			\begin{itemize}[leftmargin=1cm]
				\item Aplicación de conceptos de termodinámica estadística
				\item Análisis de distribuciones espectrales
				\item Interpretación de fenómenos cuánticos fundamentales
			\end{itemize}
			
			\textbf{Competencias Profesionales:}
			\begin{itemize}[leftmargin=1cm]
				\item Caracterización térmica de materiales
				\item Análisis de propiedades ópticas
				\item Diseño de procesos térmicos
			\end{itemize}
			
			\columnbreak
			
			\textbf{Competencias Transversales:}
			\begin{itemize}[leftmargin=1cm]
				\item Pensamiento crítico y analítico
				\item Trabajo colaborativo efectivo
				\item Comunicación científica clara
			\end{itemize}
			
			\textbf{Competencias Tecnológicas:}
			\begin{itemize}[leftmargin=1cm]
				\item Uso de simuladores científicos
				\item Análisis de datos experimentales
				\item Elaboración de gráficas técnicas
			\end{itemize}
		\end{multicols}
	\end{competenciabox}
	
	% ============================================================================
	% SECCIÓN 3: MATERIALES Y RECURSOS
	% ============================================================================
	
	\section{MATERIALES Y RECURSOS}
	
	\begin{materialesbox}
		\begin{multicols}{2}
			\textbf{Recursos Tecnológicos:}
			\begin{itemize}[leftmargin=1cm]
				\item Computadora/tablet con navegador actualizado
				\item Conexión a internet estable
				\item Simulador PhET Blackbody Spectrum
				\item Calculadora científica
			\end{itemize}
			
			\textbf{Materiales de Trabajo:}
			\begin{itemize}[leftmargin=1cm]
				\item Hoja de trabajo impresa
				\item Material para gráficas (papel milimétrico)
				\item Regla y lápices de colores
				\item Calculadora científica
			\end{itemize}
			
			\columnbreak
			
			\textbf{Recursos Digitales:}
			\begin{itemize}[leftmargin=1cm]
				\item \url{https://phet.colorado.edu/es/simulation/blackbody-spectrum}
				\item Herramientas de graficación online (Desmos)
				\item Tablas de constantes físicas
			\end{itemize}
			
			\textbf{Documentos de Apoyo:}
			\begin{itemize}[leftmargin=1cm]
				\item Manual de usuario del simulador PhET
				\item Tabla de constantes físicas fundamentales
				\item Guía de análisis estadístico de datos
				\item Formulario de leyes de radiación térmica
			\end{itemize}
		\end{multicols}
	\end{materialesbox}
	
	% ============================================================================
	% SECCIÓN 4: MARCO TEÓRICO
	% ============================================================================
	
	\section{MARCO TEÓRICO}
	
	\begin{teoriabox}
		\textbf{Conceptos Fundamentales:}
		
		Un \textbf{cuerpo negro} es un objeto idealizado que absorbe toda la radiación electromagnética incidente, independientemente de la frecuencia o ángulo. También es un emisor perfecto de radiación térmica, cuyo espectro depende únicamente de su temperatura.
		
		\textbf{Crisis de la Física Clásica:} La teoría clásica (Rayleigh-Jeans) predecía que la intensidad de radiación aumentaría indefinidamente con la frecuencia, llevando a la ``catástrofe ultravioleta'' - una predicción de energía infinita.
		
		\textbf{Solución de Planck (1900):} Propuso que la energía de los osciladores está cuantizada en múltiplos enteros de $hf$, donde $h$ es la constante de Planck y $f$ la frecuencia.
		
		\textbf{Ecuaciones Principales:}
		
		\begin{align}
			\text{Ley de Planck: } \quad & B_{\lambda}(\lambda, T) = \frac{2hc^{2}}{\lambda^{5}} \frac{1}{e^{hc/(\lambda k_{B}T)}-1} \\
			\text{Ley de Wien: } \quad & \lambda_{\text{max}} = \frac{b}{T} \quad \text{donde } b = 2.898 \times 10^{-3} \text{ m·K} \\
			\text{Ley de Stefan-Boltzmann: } \quad & P = \sigma T^4 \quad \text{donde } \sigma = 5.67 \times 10^{-8} \text{ W/m}^2\text{K}^4
		\end{align}
		
		\textbf{Conexión con Nanotecnología:}
		
		Los principios de radiación de cuerpo negro son fundamentales para:
		\begin{itemize}
			\item Caracterización térmica de nanomateriales
			\item Diseño de metamateriales con propiedades radiativas específicas
			\item Desarrollo de sensores térmicos a escala nanométrica
			\item Optimización de celdas solares y dispositivos fotovoltaicos
		\end{itemize}
	\end{teoriabox}
	
	% ============================================================================
	% SECCIÓN 5: INSTRUCCIONES GENERALES
	% ============================================================================
	
	\section{INSTRUCCIONES GENERALES}
	
	\begin{instruccionbox}
		\textbf{Antes de comenzar:}
		\begin{enumerate}
			\item Formen equipos de 3-4 integrantes y designen roles (coordinador, analista de datos, graficador, reportero)
			\item Revisen el marco teórico y asegúrense de comprender los conceptos de cuerpo negro y cuantización
			\item Verifiquen el acceso al simulador: \url{https://phet.colorado.edu/es/simulation/blackbody-spectrum}
			\item Preparen material para elaborar gráficas y registrar datos
			\item Revisen los objetivos de aprendizaje para enfocar su trabajo
		\end{enumerate}
		
		\textbf{Durante la práctica:}
		\begin{enumerate}
			\item Sigan cuidadosamente las instrucciones de cada parte, explorando sistemáticamente las variables
			\item Registren todas las observaciones y mediciones en las tablas proporcionadas
			\item Discutan los resultados en equipo antes de continuar a la siguiente parte
			\item Comparen continuamente las curvas de Planck con las de Rayleigh-Jeans
			\item Consulten dudas con el instructor cuando encuentren resultados inesperados
		\end{enumerate}
		
		\textbf{Al finalizar:}
		\begin{enumerate}
			\item Completen todas las secciones de análisis y elaboren las gráficas solicitadas
			\item Discutan las implicaciones históricas y conceptuales de sus resultados
			\item Preparen una presentación breve (3 minutos) destacando sus hallazgos principales
			\item Entreguen la hoja de trabajo completada y el reporte de conclusiones
		\end{enumerate}
	\end{instruccionbox}
	
	% ============================================================================
	% SECCIÓN 6: DESARROLLO DE LA PRÁCTICA
	% ============================================================================
	
	\section{DESARROLLO DE LA PRÁCTICA}
	
	% ----------------------------------------------------------------------------
	% PARTE 1: VERIFICACIÓN DE LA LEY DE WIEN
	% ----------------------------------------------------------------------------
	
	\subsection{PARTE 1: VERIFICACIÓN DE LA LEY DE WIEN}
	
	\subsubsection{Procedimiento:}
	\begin{enumerate}
		\item En el simulador, habiliten tanto el ``Espectro de cuerpo negro'' como la ``Curva de Rayleigh-Jeans''
		\item Configuren la temperatura inicial en 3000 K y observen la posición del máximo de emisión
		\item Para cada temperatura de la tabla, determinen con precisión la longitud de onda ($\lambda$) correspondiente al máximo de emisión
		\item Registren también la intensidad máxima (valor pico) para cada temperatura
		\item Observen cómo se comporta la curva de Rayleigh-Jeans en comparación con la de Planck
	\end{enumerate}
	
	\subsubsection{Tabla de datos:}
	
	\begin{center}
		\begin{tabular}{|c|c|c|c|}
			\hline
			\textbf{Temperatura (K)} & \textbf{$\lambda$ máximo (nm)} & \textbf{Producto $\lambda \cdot T$ (nm$\cdot$K)} & \textbf{Intensidad máxima (W/m²/nm)} \\
			\hline
			3000 & & & \\
			\hline
			4000 & & & \\
			\hline
			5000 & & & \\
			\hline
			6000 & & & \\
			\hline
			7000 & & & \\
			\hline
		\end{tabular}
	\end{center}
	
	\subsubsection{Análisis de resultados:}
	\begin{enumerate}
		\item Calculen el producto $\lambda \cdot T$ para cada temperatura
		\item Determinen si este producto es aproximadamente constante y calculen su valor promedio
		\item Comparen su valor promedio con la constante de Wien teórica ($b = 2.898 \times 10^{-3}$ m$\cdot$K)
		\item Calculen el porcentaje de error y discutan las posibles fuentes de discrepancia
	\end{enumerate}
	
	\begin{preguntabox}
		\textbf{Preguntas de Análisis - Parte 1:}
		\begin{enumerate}
			\item ¿Cómo cambia la posición del máximo del espectro al aumentar la temperatura? ¿Es esto consistente con la ley de Wien?
			
			\vspace{1.5cm}
			
			\item ¿Qué implicaciones tiene este comportamiento para el color aparente de objetos a diferentes temperaturas? Proporcione ejemplos específicos.
			
			\vspace{1.5cm}
			
			\item ¿Cómo podría utilizar la ley de Wien para estimar la temperatura de una estrella basándose en su color dominante? Explique el procedimiento.
			
			\vspace{1.5cm}
			
			\item ¿Por qué los objetos a temperatura ambiente (300 K) no son visibles en la oscuridad, pero sí detectables con cámaras termográficas?
			
			\vspace{1.5cm}
		\end{enumerate}
	\end{preguntabox}
	
	% ----------------------------------------------------------------------------
	% PARTE 2: VERIFICACIÓN DE LA LEY DE STEFAN-BOLTZMANN
	% ----------------------------------------------------------------------------
	
	\subsection{PARTE 2: VERIFICACIÓN DE LA LEY DE STEFAN-BOLTZMANN}
	
	\subsubsection{Procedimiento:}
	\begin{enumerate}
		\item Mantengan habilitada la opción ``Mostrar intensidad'' en el simulador
		\item Para cada temperatura de la tabla, anoten cuidadosamente la intensidad total (W/m²) indicada
		\item Calculen el cociente entre la intensidad total y $T^4$ para verificar la proporcionalidad
		\item Observen cómo aumenta la intensidad total al incrementar la temperatura
	\end{enumerate}
	
	\subsubsection{Tabla de datos:}
	
	\begin{center}
		\begin{tabular}{|c|c|c|}
			\hline
			\textbf{Temperatura (K)} & \textbf{Intensidad total (W/m²)} & \textbf{Intensidad/$T^4$ (W/m²/K$^4$)} \\
			\hline
			3000 & & \\
			\hline
			4000 & & \\
			\hline
			5000 & & \\
			\hline
			6000 & & \\
			\hline
			7000 & & \\
			\hline
		\end{tabular}
	\end{center}
	
	\begin{preguntabox}
		\textbf{Preguntas de Análisis - Parte 2:}
		\begin{enumerate}
			\item ¿Es el cociente Intensidad/$T^4$ aproximadamente constante? Compare con la constante teórica de Stefan-Boltzmann.
			
			\vspace{1.5cm}
			
			\item Si la temperatura de un horno industrial se duplica de 1000 K a 2000 K, ¿en qué factor aumenta la energía total radiada? Calcule y explique.
			
			\vspace{1.5cm}
			
			\item ¿Por qué es crucial la ley de Stefan-Boltzmann para aplicaciones como termografía infrarroja y diseño de hornos industriales?
			
			\vspace{1.5cm}
		\end{enumerate}
	\end{preguntabox}
	
	% ----------------------------------------------------------------------------
	% PARTE 3: ANÁLISIS DE LA CATÁSTROFE ULTRAVIOLETA
	% ----------------------------------------------------------------------------
	
	\subsection{PARTE 3: ANÁLISIS DE LA CATÁSTROFE ULTRAVIOLETA}
	
	\subsubsection{Procedimiento:}
	\begin{enumerate}
		\item Configuren la temperatura en 5000 K (aproximadamente la del Sol)
		\item Habiliten simultáneamente la curva de cuerpo negro (Planck) y la curva de Rayleigh-Jeans
		\item Observen ambas curvas, prestando especial atención a la región de longitudes de onda cortas (< 500 nm)
		\item Tomen puntos de datos donde ambas curvas difieren significativamente
		\item Analicen el comportamiento de la curva clásica a frecuencias altas
	\end{enumerate}
	
	\subsubsection{Tabla de datos:}
	
	\begin{center}
		\begin{tabular}{|c|c|c|c|}
			\hline
			\textbf{Longitud de onda (nm)} & \textbf{Intensidad Planck (W/m²/nm)} & \textbf{Intensidad Rayleigh-Jeans (W/m²/nm)} & \textbf{Ratio (R-J/Planck)} \\
			\hline
			200 & & & \\
			\hline
			400 & & & \\
			\hline
			600 & & & \\
			\hline
			1000 & & & \\
			\hline
			2000 & & & \\
			\hline
		\end{tabular}
	\end{center}
	
	\begin{preguntabox}
		\textbf{Preguntas de Análisis - Parte 3:}
		\begin{enumerate}
			\item ¿En qué región del espectro son más significativas las diferencias entre las predicciones clásica y cuántica? ¿Por qué?
			
			\vspace{1.5cm}
			
			\item Explique, en términos de la hipótesis cuántica de Planck, por qué la intensidad real disminuye a altas frecuencias.
			
			\vspace{1.5cm}
			
			\item ¿Qué implicaciones físicas tendría la predicción clásica si fuera correcta? ¿Por qué esto constituía una ``catástrofe''?
			
			\vspace{1.5cm}
			
			\item ¿Cómo resolvió la cuantización de Planck el problema de la energía infinita predicha por la teoría clásica?
			
			\vspace{1.5cm}
		\end{enumerate}
	\end{preguntabox}
	
	% ----------------------------------------------------------------------------
	% PARTE 4: ANÁLISIS COMPARATIVO DE FUENTES REALES
	% ----------------------------------------------------------------------------
	
	\subsection{PARTE 4: ANÁLISIS COMPARATIVO DE FUENTES REALES}
	
	\subsubsection{Procedimiento:}
	\begin{enumerate}
		\item Utilicen el simulador para modelar diferentes fuentes de radiación térmica
		\item Para cada fuente listada, configuren la temperatura correspondiente y registren sus características espectrales
		\item Observen en qué región del espectro electromagnético ocurre la máxima emisión
		\item Determinen qué porcentaje de la radiación es visible al ojo humano
	\end{enumerate}
	
	\subsubsection{Análisis comparativo:}
	
	\begin{center}
		\begin{tabular}{|p{3cm}|c|c|p{3.5cm}|p{2.5cm}|}
			\hline
			\textbf{Fuente} & \textbf{Temperatura (K)} & \textbf{$\lambda$ máximo (nm)} & \textbf{Región espectral dominante} & \textbf{Apariencia visual} \\
			\hline
			Temperatura ambiente & 300 & & & \\
			\hline
			Filamento de tungsteno & 3000 & & & \\
			\hline
			Sol & 5800 & & & \\
			\hline
			Estrella azul & 12000 & & & \\
			\hline
			Radiación cósmica de fondo & 2.7 & & & \\
			\hline
		\end{tabular}
	\end{center}
	
	\begin{preguntabox}
		\textbf{Preguntas de Análisis - Parte 4:}
		\begin{enumerate}
			\item ¿Por qué la radiación cósmica de fondo tiene su máximo en la región de microondas? ¿Qué significado cosmológico tiene esto?
			
			\vspace{1.5cm}
			
			\item Un material nanoestructurado tiene su pico de emisión en 900 nm. ¿Cuál sería su temperatura y cómo se vería a simple vista?
			
			\vspace{1.5cm}
			
			\item ¿Por qué las bombillas incandescentes son ineficientes energéticamente desde el punto de vista de la iluminación visible?
			
			\vspace{1.5cm}
		\end{enumerate}
	\end{preguntabox}
	
	% ============================================================================
	% SECCIÓN 7: APLICACIONES Y CASOS PRÁCTICOS
	% ============================================================================
	
	\section{APLICACIONES Y CASOS PRÁCTICOS}
	
	\subsection{Aplicación en Nanotecnología:}
	
	Los principios de radiación de cuerpo negro son fundamentales en nanotecnología para:
	
	\begin{itemize}
		\item \textbf{Caracterización térmica:} Determinación de temperaturas en procesos de síntesis de nanomateriales
		\item \textbf{Metamateriales:} Diseño de estructuras periódicas con propiedades radiativas específicas
		\item \textbf{Puntos cuánticos:} Análisis de propiedades ópticas y térmicas de nanocristales semiconductores
		\item \textbf{Sensores térmicos:} Desarrollo de detectores de radiación infrarroja a escala nanométrica
	\end{itemize}
	
	\subsection{Caso de Estudio: Puntos Cuánticos de Silicio}
	
	Los puntos cuánticos de silicio (Si-QDs) de diferentes tamaños exhiben propiedades de emisión que pueden modelarse parcialmente usando principios de cuerpo negro modificados. Investigadores han observado que:
	
	\begin{itemize}
		\item QDs de 2-3 nm emiten en el rango visible (500-700 nm)
		\item La temperatura efectiva de emisión depende del tamaño del punto cuántico
		\item Las aplicaciones incluyen celdas solares de tercera generación y displays LED
	\end{itemize}
	
	\begin{preguntabox}
		\textbf{Análisis del Caso de Estudio:}
		\begin{enumerate}
			\item ¿Cómo podrían aplicar la ley de Wien para estimar la ``temperatura efectiva'' de emisión de puntos cuánticos?
			
			\vspace{2cm}
			
			\item ¿Qué ventajas tendría un material nanoestructurado que pueda controlar su emisión térmica comparado con un cuerpo negro ideal?
			
			\vspace{2cm}
			
			\item ¿Cuáles serían las limitaciones de aplicar directamente las leyes de cuerpo negro a sistemas cuánticos nanométricos?
			
			\vspace{2cm}
		\end{enumerate}
	\end{preguntabox}
	
	% ============================================================================
	% SECCIÓN 8: EVALUACIÓN FORMATIVA
	% ============================================================================
	
	\section{EVALUACIÓN FORMATIVA}
	
	\begin{evaluacionbox}
		\textbf{Autoevaluación:} Marca con una X tu nivel de comprensión
		
		\begin{center}
			\begin{tabular}{|p{6cm}|c|c|c|c|}
				\hline
				\textbf{Criterio} & \textbf{Excelente} & \textbf{Bueno} & \textbf{Regular} & \textbf{Necesito ayuda} \\
				\hline
				Comprendo el concepto de cuerpo negro y sus propiedades & & & & \\
				\hline
				Puedo explicar la catástrofe ultravioleta y su solución & & & & \\
				\hline
				Aplico correctamente las leyes de Wien y Stefan-Boltzmann & & & & \\
				\hline
				Interpreto las diferencias entre predicciones clásicas y cuánticas & & & & \\
				\hline
				Relaciono los conceptos con aplicaciones en nanotecnología & & & & \\
				\hline
				Utilizo eficazmente el simulador PhET & & & & \\
				\hline
				Trabajo colaborativamente de manera productiva & & & & \\
				\hline
			\end{tabular}
		\end{center}
		
		\textbf{Coevaluación del Equipo:}
		\begin{itemize}
			\item ¿Cómo calificarían la colaboración y comunicación dentro del equipo? \hrulefill
			\item ¿Qué estrategias de trabajo fueron más efectivas? \hrulefill
			\item ¿Qué aspectos pueden mejorar para futuras prácticas? \hrulefill
			\item ¿Todos los miembros contribuyeron equitativamente al trabajo? \hrulefill
		\end{itemize}
	\end{evaluacionbox}
	
	% ============================================================================
	% SECCIÓN 9: CONCLUSIONES Y REFLEXIÓN
	% ============================================================================
	
	\section{CONCLUSIONES Y REFLEXIÓN}
	
	\begin{conclusionbox}
		\textbf{Síntesis de Resultados:}
		
		Elaboren un párrafo que resuma los principales hallazgos de la práctica, incluyendo los valores experimentales obtenidos para las constantes de Wien y Stefan-Boltzmann:
		
		\vspace{3cm}
		
		\textbf{Conexión con Objetivos de Aprendizaje:}
		
		¿Cómo esta práctica contribuyó al logro de los objetivos planteados? Mencionen específicamente qué conceptos ahora comprenden mejor:
		
		\vspace{3cm}
		
		\textbf{Implicaciones Históricas y Conceptuales:}
		
		Reflexionen sobre la importancia histórica del trabajo de Planck y cómo cambió nuestra comprensión de la naturaleza:
		
		\vspace{3cm}
		
		\textbf{Aplicaciones en Ingeniería en Nanotecnología:}
		
		¿Cómo pueden aplicar estos conocimientos sobre radiación térmica en su futura práctica profesional en nanotecnología?
		
		\vspace{3cm}
		
		\textbf{Reflexión Personal:}
		
		¿Qué aspecto de la práctica les resultó más interesante o desafiante? ¿Cómo cambió su perspectiva sobre la física cuántica?
		
		\vspace{3cm}
	\end{conclusionbox}
	
	% ============================================================================
	% SECCIÓN 10: ENTREGABLES Y CRITERIOS DE EVALUACIÓN
	% ============================================================================
	
	\section{ENTREGABLES Y CRITERIOS DE EVALUACIÓN}
	
	\subsection{Entregables:}
	
	\begin{enumerate}
		\item Hoja de trabajo completada con todos los datos, cálculos y análisis
		\item Gráficas de $\lambda_{max}$ vs $T$ y de Intensidad vs $T^4$ (digitales o elaboradas a mano)
		\item Reporte de conclusiones que incluya reflexiones sobre las implicaciones históricas (máximo 1 página)
		\item Presentación breve de resultados principales (3-5 minutos por equipo)
	\end{enumerate}
	
	\subsection{Criterios de Evaluación:}
	
	\begin{center}
		\begin{tabular}{|l|c|c|}
			\hline
			\textbf{Criterio} & \textbf{Puntuación máxima} & \textbf{Puntuación obtenida} \\
			\hline
			Recolección precisa y sistemática de datos & 20 puntos & \\
			\hline
			Análisis correcto de resultados y cálculos & 25 puntos & \\
			\hline
			Calidad de gráficas y presentación visual & 15 puntos & \\
			\hline
			Respuestas completas a preguntas de análisis & 25 puntos & \\
			\hline
			Conclusiones reflexivas y bien fundamentadas & 15 puntos & \\
			\hline
			\textbf{TOTAL} & \textbf{100 puntos} & \\
			\hline
		\end{tabular}
	\end{center}
	
	\begin{notabox}
		\textbf{Criterios Específicos de Calidad:}
		\begin{itemize}
			\item \textbf{Excelente (90-100):} Demuestra comprensión profunda, análisis crítico y conexiones innovadoras
			\item \textbf{Satisfactorio (70-89):} Comprende conceptos básicos y realiza análisis adecuado
			\item \textbf{Necesita mejora (<70):} Comprensión limitada o análisis incompleto
		\end{itemize}
	\end{notabox}
	
	% ============================================================================
	% SECCIÓN 11: RECURSOS COMPLEMENTARIOS
	% ============================================================================
	
	\section{RECURSOS COMPLEMENTARIOS}
	
	\subsection{Enlaces y Simuladores:}
	\begin{itemize}
		\item Simulador principal: \url{https://phet.colorado.edu/es/simulation/blackbody-spectrum}
		\item Calculadora de ley de Wien: \url{https://www.omnicalculator.com/physics/wiens-law}
		\item Graficador online: \url{https://www.desmos.com/calculator}
		\item Conversor de unidades: \url{https://www.unitconverters.net/}
	\end{itemize}
	
	\subsection{Lecturas Recomendadas:}
	\begin{itemize}
		\item Eisberg, R. \& Resnick, R. ``Física Cuántica'', Capítulo 1: Radiación Térmica y Postulado de Planck
		\item Serway, R. A. ``Física Moderna'', Capítulo 40: Introducción a la Física Cuántica
		\item Griffiths, D. ``Quantum Mechanics'', Capítulo 1: The Wave Function (secciones introductorias)
		\item Artículo: ``Max Planck and the Genesis of Quantum Theory'' - Physics Today
	\end{itemize}
	
	\subsection{Videos Educativos:}
	\begin{itemize}
		\item ``La catástrofe ultravioleta y el nacimiento de la física cuántica'' - MinutePhysics (5 min)
		\item ``Blackbody Radiation'' - Khan Academy (8 min)
		\item ``Max Planck y la revolución cuántica'' - Veritasium (12 min)
		\item ``Aplicaciones de la radiación térmica en astronomía'' - NASA Education (15 min)
	\end{itemize}
	
	\subsection{Recursos Avanzados:}
	\begin{itemize}
		\item Simulador avanzado de espectros estelares: \url{https://ccnmtl.github.io/blackbody/}
		\item Base de datos de propiedades térmicas: NIST Webbook
		\item Artículos de investigación sobre metamateriales térmicos
		\item Documentación técnica de cámaras termográficas
	\end{itemize}
	
	% ============================================================================
	% SECCIÓN 12: CONSTANTES Y FÓRMULAS DE REFERENCIA
	% ============================================================================
	
	\section{CONSTANTES Y FÓRMULAS DE REFERENCIA}
	
	\begin{notabox}
		\textbf{Constantes Físicas Fundamentales:}
		\begin{multicols}{2}
			\begin{itemize}[leftmargin=1cm]
				\item Constante de Planck: $h = 6.626 \times 10^{-34}$ J·s
				\item Velocidad de la luz: $c = 3.00 \times 10^{8}$ m/s
				\item Constante de Boltzmann: $k_B = 1.381 \times 10^{-23}$ J/K
				\item Constante de Wien: $b = 2.898 \times 10^{-3}$ m K
			%	\item Constante de Stefan-Boltzmann: $\sigma = 5.67 \times 10^{-8}$ W/m^2K^4
			\end{itemize}
			
			\columnbreak
			
			\begin{itemize}[leftmargin=1cm]
				\item Conversión útil: $hc = 1.240 \times 10^{-6}$ eV·m
				\item $k_B T$ a 300 K = 0.026 eV
				\item 1 eV = $1.602 \times 10^{-19}$ J
				\item 1 nm = $10^{-9}$ m
				\item Temperatura del Sol: $T_{\odot} = 5778$ K
			\end{itemize}
		\end{multicols}
		
		\textbf{Fórmulas Clave:}
		\begin{align}
			\text{Ley de Planck: } & \quad B_{\lambda}(\lambda, T) = \frac{2hc^{2}}{\lambda^{5}} \frac{1}{e^{hc/(\lambda k_{B}T)}-1} \\
			\text{Aproximación de Rayleigh-Jeans: } & \quad B_{\lambda}(\lambda, T) \approx \frac{2ck_{B}T}{\lambda^{4}} \quad \text{(para } \lambda \gg \frac{hc}{k_BT}\text{)} \\
			\text{Aproximación de Wien: } & \quad B_{\lambda}(\lambda, T) \approx \frac{2hc^{2}}{\lambda^{5}} e^{-hc/(\lambda k_{B}T)} \quad \text{(para } \lambda \ll \frac{hc}{k_BT}\text{)}
		\end{align}
	\end{notabox}
	
	% ============================================================================
	% APÉNDICE: GUÍA RÁPIDA DEL SIMULADOR
	% ============================================================================
	
	\section{APÉNDICE: GUÍA RÁPIDA DEL SIMULADOR PhET}
	
	\subsection{Controles Principales:}
	\begin{itemize}
		\item \textbf{Deslizador de temperatura:} Permite ajustar la temperatura del cuerpo negro (250 K - 50,000 K)
		\item \textbf{Checkbox ``Curva de cuerpo negro'':} Muestra/oculta la distribución de Planck
		\item \textbf{Checkbox ``Curva de Rayleigh-Jeans'':} Muestra/oculta la predicción clásica
		\item \textbf{Checkbox ``Intensidad'':} Muestra el valor numérico de la intensidad total
		\item \textbf{Zoom:} Permite ampliar regiones específicas del espectro
	\end{itemize}
	
	\subsection{Consejos de Uso:}
	\begin{itemize}
		\item Use el cursor para leer valores específicos en las curvas
		\item Experimente con temperaturas extremas para observar comportamientos límite
		\item Compare siempre las predicciones clásicas con las cuánticas
		\item Tome capturas de pantalla para incluir en su reporte
	\end{itemize}
	
	\subsection{Interpretación de Resultados:}
	\begin{itemize}
		\item A altas temperaturas: El máximo se desplaza hacia longitudes de onda más cortas
		\item En el límite clásico: Las curvas de Planck y Rayleigh-Jeans convergen a bajas frecuencias
		\item Catástrofe ultravioleta: La curva clásica diverge a altas frecuencias
		\item Cuantización: La curva de Planck tiende a cero a altas frecuencias
	\end{itemize}
	
	% ============================================================================
	% PROBLEMAS ADICIONALES PARA PRÁCTICA EXTRA
	% ============================================================================
	
	\section{PROBLEMAS ADICIONALES (OPCIONAL)}
	
	Para estudiantes que terminen temprano o deseen profundizar más:
	
	\begin{enumerate}
		\item Calcule la temperatura de una estrella que tiene su máximo de emisión en 700 nm (estrella roja).
		
		\item Determine qué porcentaje de la radiación del Sol (T = 5778 K) es emitida en el rango visible (400-700 nm).
		
		\item Un horno de nanotubos de carbono opera a 2500 K. ¿Cuánta energía por metro cuadrado emite por segundo?
		
		\item Estime la temperatura de la radiación cósmica de fondo si su máximo está en 1.9 mm.
		
		\item Compare la eficiencia luminosa (porcentaje de energía emitida como luz visible) de una bombilla incandescente (2800 K) con una lámpara halógena (3200 K).
	\end{enumerate}
	
	% ============================================================================
	% PIE DE PÁGINA
	% ============================================================================
	
	\vfill
	
	\begin{center}
		\hrule
		\vspace{0.3cm}
		\textcolor{uteqgray}{\textit{Universidad Tecnológica de Querétaro | Manual de Prácticas de Laboratorio}}\\
		\textcolor{uteqgray}{\textit{Física Moderna | Cuatrimestre Mayo-Agosto 2025}}\\
		\textcolor{uteqgray}{\textit{``La cuantización de Planck: El nacimiento de una nueva era en la física''}}
	\end{center}
	
\end{document}