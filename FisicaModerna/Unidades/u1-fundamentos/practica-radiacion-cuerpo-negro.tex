\documentclass[12pt,a4paper]{article}
\usepackage[utf8]{inputenc}
\usepackage[spanish]{babel}
\usepackage{amsmath}
\usepackage{amsfonts}
\usepackage{amssymb}
\usepackage{graphicx}
\usepackage{geometry}
\usepackage{hyperref}
\usepackage{array}
\usepackage{multirow}
\usepackage{booktabs}
\usepackage{fancyhdr}
\usepackage{xcolor}
\usepackage{tcolorbox}

% Configuración de página
\geometry{margin=2.5cm}
\pagestyle{fancy}
\fancyhf{}
\fancyhead[L]{Física Moderna - Ingeniería en Nanotecnología}
\fancyhead[R]{Universidad Tecnológica de Querétaro}
\fancyfoot[C]{\thepage}

% Configuración de colores
\definecolor{uteqblue}{RGB}{0,51,102}
\definecolor{uteqgray}{RGB}{128,128,128}

% Configuración de cajas de texto
\newtcolorbox{objetivobox}{
	colback=blue!5!white,
	colframe=uteqblue,
	title=Objetivos,
	fonttitle=\bfseries
}

\newtcolorbox{instruccionbox}{
	colback=gray!5!white,
	colframe=uteqgray,
	title=Instrucciones,
	fonttitle=\bfseries
}

\newtcolorbox{preguntabox}{
	colback=orange!5!white,
	colframe=orange!50!black,
	title=Preguntas de Análisis,
	fonttitle=\bfseries
}

\newtcolorbox{reflexionbox}{
	colback=green!5!white,
	colframe=green!50!black,
	title=Reflexión,
	fonttitle=\bfseries
}

\begin{document}
	
	% Título principal
	\begin{center}
		{\Huge \textbf{ACTIVIDAD PRÁCTICA}}\\[0.3cm]
		{\LARGE \textbf{RADIACIÓN DE CUERPO NEGRO}}\\[0.2cm]
		{\large Física Moderna | Ingeniería en Nanotecnología}\\[0.5cm]
		\includegraphics[width=5cm]{../../Imagenes/Logo_uteq}
	\end{center}
	
	\vspace{0.5cm}
	
	% Datos generales
	\section*{DATOS GENERALES}
	\begin{itemize}
		\item \textbf{Duración:} 45 minutos
		\item \textbf{Modalidad:} Trabajo colaborativo en equipos de 3-4 estudiantes
		\item \textbf{Materiales:} Computadora/tablet con acceso a internet, calculadora, hoja de trabajo impresa
	\end{itemize}
	
	% Objetivos
	\begin{objetivobox}
		Al finalizar esta actividad, el estudiante será capaz de:
		\begin{itemize}
			\item Analizar experimentalmente la distribución espectral de la radiación de cuerpo negro a diferentes temperaturas
			\item Verificar las leyes de Wien y Stefan-Boltzmann utilizando datos de simulación
			\item Determinar la constante de Planck mediante análisis gráfico
			\item Comparar las predicciones de la física clásica con el modelo cuántico de Planck
		\end{itemize}
	\end{objetivobox}
	
	\section*{INTRODUCCIÓN}
	
	La simulación PhET ``Radiación de Cuerpo Negro'' permite explorar el comportamiento del espectro de emisión térmica a diferentes temperaturas. En esta actividad, utilizaremos este simulador para verificar las leyes fundamentales de la radiación térmica y analizar cómo la cuantización de la energía de Planck resolvió la ``catástrofe ultravioleta'' que predecía la física clásica.
	
	% Instrucciones generales
	\begin{instruccionbox}
		\begin{enumerate}
			\item Formen equipos de 3-4 integrantes
			\item Accedan al simulador PhET: \url{https://phet.colorado.edu/es/simulation/blackbody-spectrum}
			\item Completen cada una de las secciones siguientes, registrando sus observaciones y resultados
			\item Discutan las preguntas de análisis en equipo
			\item Preparen una breve presentación (2-3 minutos) con sus conclusiones principales
		\end{enumerate}
	\end{instruccionbox}
	
	\newpage
	
	\section{PARTE 1: VERIFICACIÓN DE LA LEY DE WIEN}
	
	\subsection{Procedimiento:}
	\begin{enumerate}
		\item En el simulador, muestra tanto el ``Espectro de cuerpo negro'' como la ``Curva de Rayleigh-Jeans''
		\item Para cada temperatura de la tabla, determina la longitud de onda ($\lambda$) correspondiente al máximo de emisión
		\item Registra también la intensidad máxima (valor pico) para cada temperatura
	\end{enumerate}
	
	\subsection{Tabla de datos:}
	
	\begin{center}
		\begin{tabular}{|c|c|c|c|}
			\hline
			\textbf{Temperatura (K)} & \textbf{$\lambda$ máximo (nm)} & \textbf{Producto $\lambda \cdot T$ (nm$\cdot$K)} & \textbf{Intensidad máxima (W/m²/nm)} \\
			\hline
			3000 & & & \\
			\hline
			4000 & & & \\
			\hline
			5000 & & & \\
			\hline
			6000 & & & \\
			\hline
			7000 & & & \\
			\hline
		\end{tabular}
	\end{center}
	
	\subsection{Análisis:}
	\begin{enumerate}
		\item Calculen el producto $\lambda \cdot T$ para cada temperatura
		\item ¿Es este producto aproximadamente constante? ¿Cuál es su valor promedio?
		\item Comparen su valor promedio con la constante de Wien teórica ($b = 2.898 \times 10^{-3}$ m$\cdot$K)
		\item Calculen el porcentaje de error
	\end{enumerate}
	
	\begin{preguntabox}
		\textbf{Preguntas:}
		\begin{enumerate}
			\item ¿Cómo cambia la posición del máximo del espectro al aumentar la temperatura?
			\item ¿Qué implicaciones tiene esto para el color aparente de objetos a diferentes temperaturas?
			\item ¿Podría utilizar esta ley para estimar la temperatura de una estrella basándose en su color? Explique.
		\end{enumerate}
	\end{preguntabox}
	
	\section{PARTE 2: VERIFICACIÓN DE LA LEY DE STEFAN-BOLTZMANN}
	
	\subsection{Procedimiento:}
	\begin{enumerate}
		\item Mantengan habilitada la opción ``Mostrar intensidad''
		\item Para cada temperatura de la tabla, anoten la intensidad total (W/m²) indicada en el simulador
		\item Calculen el cociente entre la intensidad total y $T^4$
	\end{enumerate}
	
	\subsection{Tabla de datos:}
	
	\begin{center}
		\begin{tabular}{|c|c|c|}
			\hline
			\textbf{Temperatura (K)} & \textbf{Intensidad total (W/m²)} & \textbf{Intensidad/$T^4$ (W/m²/K$^4$)} \\
			\hline
			3000 & & \\
			\hline
			4000 & & \\
			\hline
			5000 & & \\
			\hline
			6000 & & \\
			\hline
			7000 & & \\
			\hline
		\end{tabular}
	\end{center}
	
	\subsection{Análisis:}
	\begin{enumerate}
		\item ¿Es el cociente Intensidad/$T^4$ aproximadamente constante?
		\item Calculen el valor promedio de este cociente
		\item Comparen su valor con la constante de Stefan-Boltzmann teórica ($\sigma = 5.67 \times 10^{-8}$ W/m²K$^4$)
		\item Calculen el porcentaje de error
	\end{enumerate}
	
	\begin{preguntabox}
		\textbf{Preguntas:}
		\begin{enumerate}
			\item Si la temperatura de un objeto se duplica, ¿en qué factor aumenta la intensidad total radiada?
			\item ¿Por qué es tan importante la ley de Stefan-Boltzmann para aplicaciones como la termografía?
			\item Un horno industrial opera a 1500 K. Si se incrementa su temperatura a 1800 K, ¿en qué porcentaje aumentará la energía radiada?
		\end{enumerate}
	\end{preguntabox}
	
	\newpage
	
	\section{PARTE 3: LA CATÁSTROFE ULTRAVIOLETA}
	
	\subsection{Procedimiento:}
	\begin{enumerate}
		\item Configurar la temperatura en 5000 K
		\item Habilitar tanto la curva de cuerpo negro (Planck) como la curva de Rayleigh-Jeans
		\item Observar ambas curvas, particularmente en la región de longitudes de onda cortas
		\item Tomar varios puntos donde ambas curvas difieren significativamente
	\end{enumerate}
	
	\subsection{Tabla de datos:}
	
	\begin{center}
		\begin{tabular}{|c|c|c|c|}
			\hline
			\textbf{Longitud de onda (nm)} & \textbf{Intensidad Planck (W/m²/nm)} & \textbf{Intensidad Rayleigh-Jeans (W/m²/nm)} & \textbf{Ratio (R-J/Planck)} \\
			\hline
			200 & & & \\
			\hline
			400 & & & \\
			\hline
			600 & & & \\
			\hline
			1000 & & & \\
			\hline
			2000 & & & \\
			\hline
		\end{tabular}
	\end{center}
	
	\subsection{Análisis:}
	\begin{enumerate}
		\item ¿En qué región del espectro las diferencias entre las predicciones clásica y cuántica son más significativas?
		\item ¿Qué sucede con la curva de Rayleigh-Jeans a longitudes de onda muy cortas?
		\item En la región de longitudes de onda largas, ¿convergen ambas curvas? ¿Por qué?
	\end{enumerate}
	
	\begin{preguntabox}
		\textbf{Preguntas:}
		\begin{enumerate}
			\item Explique, en términos de la hipótesis cuántica de Planck, por qué la intensidad real disminuye a altas frecuencias.
			\item ¿Por qué se denomina ``catástrofe ultravioleta'' al problema de la predicción clásica?
			\item ¿Qué implicaciones físicas tendría la predicción clásica si fuera correcta?
		\end{enumerate}
	\end{preguntabox}
	
	\section{PARTE 4: ANÁLISIS COMPARATIVO DE ESPECTROS}
	
	\subsection{Procedimiento:}
	\begin{enumerate}
		\item Configurar el simulador para mostrar simultáneamente espectros a diferentes temperaturas
		\item Observar y registrar el espectro para las siguientes fuentes:
	\end{enumerate}
	
	\subsection{Análisis comparativo:}
	
	\begin{center}
		\begin{tabular}{|p{3cm}|c|c|p{3.5cm}|p{2.5cm}|}
			\hline
			\textbf{Fuente} & \textbf{Temperatura (K)} & \textbf{$\lambda$ máximo (nm)} & \textbf{Región espectral dominante} & \textbf{Apariencia visual} \\
			\hline
			Temperatura ambiente & 300 & & & \\
			\hline
			Filamento de tungsteno & 3000 & & & \\
			\hline
			Sol & 5800 & & & \\
			\hline
			Estrella azul & 12000 & & & \\
			\hline
			Radiación cósmica de fondo & 2.7 & & & \\
			\hline
		\end{tabular}
	\end{center}
	
	\subsection{Análisis:}
	\begin{enumerate}
		\item ¿Qué porcentaje de la radiación a temperatura ambiente es visible para el ojo humano?
		\item ¿Por qué no podemos ``ver'' el calor emitido por objetos a temperatura ambiente?
		\item ¿Cómo se relaciona la temperatura de una estrella con su color?
	\end{enumerate}
	
	\begin{preguntabox}
		\textbf{Preguntas:}
		\begin{enumerate}
			\item ¿Por qué la radiación cósmica de fondo tiene su máximo en la región de microondas?
			\item Un objeto tiene su pico de emisión en aproximadamente 900 nm. ¿De qué color se vería y cuál sería su temperatura aproximada?
			\item ¿Por qué el filamento de las bombillas incandescentes debe alcanzar temperaturas tan altas para ser eficiente?
		\end{enumerate}
	\end{preguntabox}
	
	\newpage
	
	\section{PARTE 5: APLICACIONES TECNOLÓGICAS Y CIENTÍFICAS}
	
	\subsection{Análisis de aplicaciones:}
	
	Para cada una de las siguientes aplicaciones, identifiquen qué aspecto de la radiación de cuerpo negro es relevante y cómo se utiliza:
	
	\subsubsection{1. Termografía infrarroja:}
	\begin{itemize}
		\item Principio físico utilizado: \hrulefill
		\item Rango de temperaturas típico: \hrulefill
		\item Ventajas y limitaciones: \hrulefill
	\end{itemize}
	
	\subsubsection{2. Determinación de la temperatura de estrellas:}
	\begin{itemize}
		\item Método utilizado: \hrulefill
		\item Información adicional que se puede obtener: \hrulefill
		\item Limitaciones del método: \hrulefill
	\end{itemize}
	
	\subsubsection{3. Diseño de hornos industriales y procesos térmicos:}
	\begin{itemize}
		\item Consideraciones importantes basadas en la radiación térmica: \hrulefill
		\item Optimización energética: \hrulefill
		\item Aplicaciones en nanotecnología: \hrulefill
	\end{itemize}
	
	\subsubsection{4. Radiación cósmica de fondo:}
	\begin{itemize}
		\item Significado cosmológico: \hrulefill
		\item Temperatura actual y su significado: \hrulefill
		\item ¿Cómo se determinó que era radiación de cuerpo negro?: \hrulefill
	\end{itemize}
	
	\section{PARTE 6: IMPLICACIONES CONCEPTUALES E HISTÓRICAS}
	
	\begin{reflexionbox}
		\textbf{Reflexión:}
		
		Discutan en grupo y elaboren una respuesta consensuada para cada una de las siguientes preguntas:
		
		\begin{enumerate}
			\item ¿Por qué la solución al problema del cuerpo negro marcó el inicio de la física cuántica?
			
			\vspace{1.5cm}
			
			\item ¿Qué tuvo que ``sacrificar'' Planck de la física clásica para resolver el problema de la radiación del cuerpo negro?
			
			\vspace{1.5cm}
			
			\item ¿Cómo afectó la hipótesis cuántica de Planck a la visión mecanicista del mundo que prevalecía en la física hasta ese momento?
			
			\vspace{1.5cm}
			
			\item ¿Qué otras teorías o descubrimientos fueron posibles gracias a la introducción del concepto de cuantización de la energía?
			
			\vspace{1.5cm}
		\end{enumerate}
	\end{reflexionbox}
	
	\section*{CONCLUSIONES Y RESULTADOS}
	
	Elaboren un resumen de los principales hallazgos y conclusiones de la actividad:
	
	\begin{enumerate}
		\item Valor experimental obtenido para la constante de Wien: \hrulefill
		\item Valor experimental obtenido para la constante de Stefan-Boltzmann: \hrulefill
		\item Principal diferencia entre la predicción clásica y la cuántica: \hrulefill
		\item Aplicación más relevante de la radiación de cuerpo negro en el campo de la nanotecnología: \hrulefill
		\item Reflexión final sobre la importancia histórica y conceptual del trabajo de Planck: 
		
		\vspace{2cm}
	\end{enumerate}
	
	\section*{ENTREGABLES}
	
	Al finalizar la actividad, cada equipo debe entregar:
	\begin{enumerate}
		\item Esta hoja de trabajo completada con todos los datos y respuestas
		\item Gráficas elaboradas (pueden ser hechas a mano o en computadora)
		\item Un breve informe con sus conclusiones (máximo una página)
	\end{enumerate}
	
	\section*{CRITERIOS DE EVALUACIÓN}
	
	\begin{center}
		\begin{tabular}{|l|c|}
			\hline
			\textbf{Criterio} & \textbf{Puntuación máxima} \\
			\hline
			Recolección precisa de datos & 20 puntos \\
			\hline
			Cálculos y análisis correctos & 25 puntos \\
			\hline
			Gráficas bien elaboradas & 15 puntos \\
			\hline
			Respuestas a preguntas de análisis & 25 puntos \\
			\hline
			Conclusiones y reflexiones & 15 puntos \\
			\hline
			\textbf{Total} & \textbf{100 puntos} \\
			\hline
		\end{tabular}
	\end{center}
	
	\section*{RECURSOS DE APOYO}
	
	\begin{itemize}
		\item Simulador PhET: \url{https://phet.colorado.edu/es/simulation/blackbody-spectrum}
		\item Constante de Wien: $b = 2.898 \times 10^{-3}$ m$\cdot$K
		\item Constante de Stefan-Boltzmann: $\sigma = 5.67 \times 10^{-8}$ W/m²K$^4$
		\item Constante de Planck: $h = 6.626 \times 10^{-34}$ J$\cdot$s
		\item Constante de Boltzmann: $k = 1.381 \times 10^{-23}$ J/K
		\item Velocidad de la luz: $c = 3.00 \times 10^{8}$ m/s
	\end{itemize}
	
	\vfill
	
	\begin{center}
		\textcolor{uteqgray}{\textit{Universidad Tecnológica de Querétaro | Física Moderna | Cuatrimestre Mayo-Agosto 2025}}
	\end{center}
	
\end{document}